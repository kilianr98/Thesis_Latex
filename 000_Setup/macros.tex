
%#############################################################################
%
% Makros       Arnemann
%
% 2015-10-18ar
% 2016-03-02 
%%=============================================================================
%%-- Farben 
%%=============================================================================
%%
%%% Farbe .s http://www.namsu.de/Extra/pakete/Xcolor.html
%%%  Hochschulfarbe
%\definecolor{hskarot}{rgb}{219,0,49} 
%\definecolor{hskarot}{rgb}{0.86,0,0.192} % 
\definecolor{hskarot}{rgb}{0.7,0,0} % kräftig rot Ersatz für Hska orig

%\colorlet{showcolor}{white}       % zum Verbergen von Texten  z.B. in ToDo-boxen
                                   % Text wird aber gedruckt, man kann danach suchen 
%\colorlet{showcolor}{red}          % zum Verbergen von Texten  z.B. in ToDo-boxen
%
%%-----------------------------------------------------------------------------
%% Neue Umgebungen
%%-----------------------------------------------------------------------------
%
% --- Gleichungen 
% Syntax: \beq{NAME DER GLEICHUNG} 
%         \eeq
% Referenz: \ref{eqt:NAME DER GLEICHUNG}
%
\newcommand{\beq}[1]
           {
            \begin{equation}
            \label{#1}
           }
% end
\newcommand{\eeq}
           {
             \end{equation}
           }
\newcommand{\ba}{\begin{array}}
\newcommand{\ea}{\end{array}}

\newcommand{\bdm}{\begin{displaymath}}
\newcommand{\edm}{\end{displaymath}}
	   
% --- Itemize 
\newcommand{\bi}{\begin{itemize}}
\newcommand{\ei}{\end{itemize}}

% --- Itemize 
\newcommand{\bci}{\begin{compactitem}}
\newcommand{\eci}{\end{compactitem}}


% --- Enumerate
\newcommand{\be}{\begin{enumerate}}
\newcommand{\ee}{\end{enumerate}}

% ---
\newcommand{\bd}{\begin{description}}
\newcommand{\ed}{\end{description}}

% --- Umgebungen wie compactitem, compactenum etc
% erlauben kleinen Abstand zwischen Aufzählungen


% --- Umgebung
%\def\d@nger{\marginpar[\hfill\dbend]{\dbend\hfill}}
%\newenvironment{danger}{\medskip\hspace{0pt}\d@nger}{\medskip}
%--- Umgebung  2015-05-09  diese Definition mit aen ist besser als mit @ . Sonst erhebl Probleme
\def\daenger{\marginpar[\hfill\dbend]{\dbend\hfill}}
\newenvironment{danger}{\medskip\hspace{0pt}\daenger}{\medskip}

%%-----------------------------------------------------------------------------
%% Abkürzungen 
%%-----------------------------------------------------------------------------
%
%--- Fußnoten
% Syntax: \fn legt fest, wo das Fußnotenzeichen steht
%         \fnt{FUSSNOTENTEXT} legt den Text fest
\newcommand{\fn}{\footnotemark}
\newcommand{\fnt}[1]{\footnotetext{#1}}

\newcommand{\ol}{\ddot{O}l}
\newcommand{\p}{\partial}
\newcommand{\Dp}{\Delta p}
\newcommand{\te}{$\vartheta$}             % spezielles theta für Temperatur Celsius
\newcommand{\cT}{\vartheta}               %      Celsiustemperatur, dass nicht in  t geändert werden darf
\newcommand{\Ct}{\vartheta}               %var   Celsiustemperatur, kann evtl in t geändert werden
\newcommand{\R}{{\em\bf R}}               % Universelle Gaskonstante fett 
%%
%% --- Einheiten werden vorzugsweise mit dem package siunitx formatiert !
%%
\newcommand{\C}{$^\circ$C}                % Grad Celsius als Einheit im Text
%\newcommand{\C}{~\textcentigrade{}}      % Grad Celsius alternativ
%\newcommand{\CC}{^\circ\mbox{C}}         % Grad Celsius alternativ2
\newcommand{\CC}{\,^\circ\mathrm{C}}      % Grad Celsius als Einheit im mathemodus
\newcommand{\CCe}{^\circ\mathrm{C}}       % Grad Celsius als Einheit im mathemodus
%%
%% MatheOperatoren
%%
\newcommand{\mue}{\textmu}                % 
\renewcommand{\d}{\partial\mspace{2mu}}   % partielles Diff. Zeichen 
\newcommand{\td}{\,\mathrm{d}}           	% totales Diff (d, nicht kursiv)
\newcommand{\ddt}[1]{\frac{\td #1}{\td t}}% zweifach 
%% siehe
%% from physic  shttp://www.dfcd.net/articles/latex/latex.html
%%

% ---  unbestimmtes Differenzial  (kleines d mit horizontalem Strich durch
\def\dbar{{\mathchar'26\mkern-11mu\mathrm{d}}}  % hier für lmodern,  Achtung passt nich bei jedem Font !
%\def\dbar{{\mathchar'26\mkern-12mu d}}  %(The space after the \mu" is optional but is added f
%\def\dbar{{\mathchar'26\mkern-9mu\mathrm{d}}}  % 130102ar hier für times !
%%%
%% Index: oben oder/und unten
%%
\newcommand{\idx}[1]{_\mathrm{#1}}        % nicht kursiver Index in Gleichungen, mit Umlauten wie "a
\newcommand{\idxi}[2]{_{\mathrm{#1,}{#2}}}% 1. Parameter nicht kursiv, zweiter kursivzB für Laufvariable
\newcommand{\idy}[1]{^\mathrm{#1}}        % hochgestellt

%%
%% Texthervorhebungen
%%
\newcommand{\name}[1]{\textsc{#1}}        % für Firmen, Autoren
\newcommand{\uu}[1]{\emph{#1}}            % Texthervorhebung (Unterstreichungen sind unüblich)%%
\newcommand{\fettA}[1]{{\sffamily\small\textbf{#1}}}   % Texthervorhebung fett

\newcommand{\s}{\scriptscriptstyle}
\newcommand{\D}{\displaystyle}            % für Fonts in Brüchen

\newcommand{\bff}[1]{\noindent {\textbf{#1}}}

\newcommand{\du}[1]{\underline{\underline{#1}}} %  Ergebnisse von Berechungen doppelt unterstreichen

%%
%% Abkürzungen im Text, alphabetisch, 
%%
\def\bzw{bzw.\ }
\def\bspw{bspw.\ }
\def\ca{ca. }
%\def\CO2{CO$_2$}  % neu
\def\dh{d.\,h.\ }
\def\etc{etc.\ }
\def\evtl{evtl. }
\def\ggf{ggf.\ }
\def\inkl{inkl.\ }
\def\oä{o.\,ä.\ }
\def\og{o.\,g.\ }
\def\so{s.\,o.}
\def\su{s.\,u.}
\def\ua{u.\,a.\ }
\def\zB{z.\,B.\ }
\def\zT{z.\,T.\ }

%%
%%---------------------------------------------------------------
%%   Abkürzungen für Variablen in Gleichungen
%%---------------------------------------------------------------
%%
%% verkürzte Schreibweise, mathematische Symbole, alphabetisch, Beispiele
%%   wichtige Regel: keine Zahlen in Variablennamen 
%% 
\def\mp{\dot{m}}
\def\md{\dot{m}}                      %
%\def\m.{\dot{m}} % geht?
\def\pWsT{p\idx{W}\idy{S}(T)}         % 
\def\Vd{\dot{V}}                      %  

\def\hepx{h_\mathrm{1+X}}             % 

%%
%% Kennzahlen werden auch kursiv formatiert, aber mit wenig Zwischenraum, darum so:
%%  muss in $ $ 

\def\GWP{\mathin{GWP}}  % 
\def\Re{\mathin{Re}}    % Reynold
\def\Nu{\mathin{Nu}}    % Nusselt
\def\Pr{\mathin{Pr}}    % Prandtl
%%
%% --- Referenzen im Text -----------------------------------------------
%%
%  Referenzieren eine Tabelle im Text, beim ersten Mal
\newcommand{\RefTab}[1]{{\small \color{LinkColor}\raisebox{1pt}{$\blacktriangleright$}}{Tabelle~\ref{#1}}} 
%\newcommand{\RefTab}[1]{$\underline{\mbox{Tab.~\ref{#1}}}$}   
%\newcommand{\RefTab}[1]{\textbf{Tabelle~\ref{#1}}}            
%
%
%  Referenzieren eine Abbildung im Text: sollte mit RefTab identisch sein
\newcommand{\RefFig}[1]{{\small \color{LinkColor}\raisebox{1pt}{$\blacktriangleright$}}{Bild~\ref{#1}}} 
%\newcommand{\RefFig}[1]{{\color{blue}$\blacktriangleright$\textbf{Bild~\ref{#1}}}} 
%\newcommand{\RefFig}[1]{\textbf{Bild~\ref{#1}}}                 
%
%  Referenzieren eine Tabelle im Text, nach dem ersten Mal: sollte nicht hervorgehoben sein
\newcommand{\RefTabc}[1]{Tabelle~\ref{#1}}                     % 2. Tabellenref. im Text
\newcommand{\RefFigc}[1]{Bild~\ref{#1}}                        % 2. Bildref.
%
%  Referenzieren einer Gleichung 
\newcommand{\RefEq}[1]{\mbox{Gl.~(\ref{#1})}}                  % Gleichungen

%%
%% --- Einheiten werden immer steil formatiert! 
%%     Einheiten werden vorzugsweise mit dem *package siunits* formatiert !
%%     folgende Abkürzungen benötigen das Package nicht 
%%
\newcommand{\ZmE}[2]{$#1$~{#2}}                                % Zahl:#1 mit Einheiten#2 #3 in TextUmgebung
\newcommand{\zme}[2]{#1~\mbox{#2}}                             % Zahl:#1 mit Einheiten#2 #3 in MatheUmgebung
\newcommand{\EH}[1]{#1}                                        % Einheiten
\newcommand{\B}[1]{\mbox{#1}}                                  % Steiler Text in MatheUmgebung 
\newcommand{\eb}[2]{\frac{\text{#1}}{\text{#2}}}               % EINHEITEN in echten Brüchen

%%
%%--- Font Größe  ----------------------------------------------------------
%%
\newfont{\ssf}{cmss10 scaled 1000}
\newfont{\ssb}{cmssbx10 scaled 1000}
\newcommand{\cf}{\ssf}                 % CaptionFonts: in Bildunter-, Tab-Überschriften
\newcommand{\rf}{\em}                  % RefFonts    : Kennzeichnung von Referenzen im Text
\newcommand{\eng}{\ttfamily}           % englische Begriffe im deutschen Text besonders hervorheben
%
% Hinweis
% Syntax: \oops{ÜBERSCHRIFT}{TEXT}
%        : nach Überschtift wird automatisch eingefügt
\newcommand{\oops}[2]{\begin{quote}\textbf{#1}:\\ {#2} \end{quote} }

%---------------------------------------------------------------
% Bildunter- und Tabellenüberschriften
%---------------------------------------------------------------
% z.B. andere Schrift, oder auch Schriftform und andere Abkürzung
%
% alternaiv : "Abbildungen" lassen und
%  Zeilenumbruch bei Bildbeschreibungen einführen \setcapindent{1em}
% bessere Methoden: siehe Komaskript
%
\addto\captionsngerman{% 
\renewcommand{\figurename}{Bild}   % Standard wäre Abbildung
\renewcommand{\tablename}{Tabelle}    % Standard wäre Tabelle
} 
%
\renewcommand*{\captionformat}{.~} % DIN lässt grüssen
%\addtokomafont{caption}{\sffamily\small\raggedright}  % kleinere Schrift, linksbündig
\addtokomafont{caption}{\sffamily\small}  % kleinere Schrift in Überschrift
%\setkomafont{descriptionlabel}{\sffamily\small}
\setkomafont{captionlabel}{\sffamily\bfseries}
\setcaphanging
\setcapindent{0em}             % kein Einzug
%
%---------------------------------------------------------------
%  Kommentare
%---------------------------------------------------------------
%-- Formatieren eigener Kommentare, Anmerkungen, Arbeitsanweisungen 
\newcommand{\Kommentar}[1]{{\em #1}} 

%---------------------------------------------------------------
%-- Verstecktes
% Alles innerhalb von \Hide{} oder \ignore{}
% wird von LaTeX komplett ignoriert (wie ein Kommentar)
%------------------------
% methode 1
\newcommand{\Hide}[1]{}
\let\ignore\Hide
%
% methode 2
%\newcommand{\Hide}[1]{
%{\color{cyan}\sffamily\small #1}} % green red blue magenta cyan
%
% methode 3
\newcommand{\HideB}[1]{}
\let\ignore\HideB

%-----------------------------------------------------
% auskommentieren von längeren Textpassgen 
%
%\newcommand{\showme}{1}  % \usepackage{ifthen}
\newcommand{\hideandshow}[1]{%
	\ifthenelse{\isundefined{\showme}}{}{#1}}


%--- Umbrüche
%  eigner Befehl, lässt sich besser modifizieren
%  eigene Umbrüche, die nur bei der Textentwicklung eingeführt wurden,
%  können hier zentral entfernt werden
\newcommand{\mynewpage}{\newpage}
\newcommand{\myclearpage}{\clearpage}

%%====================================================================
%%
%%--  Tabellen 
%%
%\usepackage{tabularx} % automatische Spaltenbreite 
\newcolumntype{L}[1]{>{\raggedright\arraybackslash}p{#1}}
\newcolumntype{C}[1]{>{\centering\arraybackslash}p{#1}}
\newcolumntype{R}[1]{>{\raggedleft\arraybackslash}p{#1}}

\newcommand{\LCIRC}[1]{{\large\textcircled{\normalsize \texttt{#1}}}}
\newcommand{\NCIRC}[1]{\textcircled{ {\small \texttt{#1}} } }

\newcommand{\bul}{$\bullet$}              % Mark in Tabs
\newcommand{\Q}{$\bullet$}                % mark2 in Tabs

\newcommand{\mc}{\multicolumn}
\newcommand{\cg}{\centering}    % for multicolumn


%%%%%%%%%%%%%%%%%%%%%%%%%%%%%%%%%%%%%%%%%%%%%%%%%%%%%%%%%%%%%%%%%%%%%%%%%%%%%%%%%%%%%
%%% EOF  %%%%%%%%%%%%%%%%%%%%%%%%%%%%%%%%%%%%%%%%%%%%%%%%%%%%%%%%%%%%%%%%%%%%%%%%%%%%
%%%%%%%%%%%%%%%%%%%%%%%%%%%%%%%%%%%%%%%%%%%%%%%%%%%%%%%%%%%%%%%%%%%%%%%%%%%%%%%%%%%%%
