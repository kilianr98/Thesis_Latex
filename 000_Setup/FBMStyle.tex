%==========================================================================================
%%
%% This is file 'FBMStyle.tex'  v 1.0
%% basieren auf einem Muster von Uwe Kappler 
%%
%%  2016-03-25 
%%========================================================================================== 
\documentclass[
        final,                % wenn es denn fertig ist.. 
        11pt,                 % Schriftgröße (10pt, 11pt, 12pt) 11:pala:ok, 
        twoside,              % oneside, twoside
        numbers=noenddot,     % Kapitelnummer ohne . am Ende
        headings=normal,      % Groesse der Ueberschrift (bigheadings, normalheadings, 
        parskip=half,         % Europäischer Satz mit Abstand zwischen Absätzen
        index=totoc,          % Index ins Verzeichnis einfügen	*** prüfen     
%        headsepline,          % Strich unter Kopfzeile
        headinclude,          % Kopfzeile bei Satzspiegel bereucksichtigen
        DIV=15,                % mit palatino 11pt oder CM 12
        BCOR=14mm]             % Binderand    %1mm, für Klebebindung % 14mm dann etwa gleichmäßig
   {scrbook}                  % Dokumenttyp // für große Berichte, Proejektarbeitebn, Abschlussarbeiten
%----------------------------------------------------------------------------------------
%  PACKAGES
%----------------------------------------------------------------------------------------
\typearea                  % nach der Schrift
        [current]          % Heftrand BCOR oben definiert
%        {calc}             % DIV neu berechnen aus package 
        {last}             % DIV letzte aus Definition zuvor

%XXXXXXXXXXXXXXXXxx
%TEST
%    XXXXXXXXXXX    
\setcounter{tocdepth}{2}
\setcounter{secnumdepth}{3}

\usepackage{lmodern}          % empfohlen anstatt CM Fonts, nur so korrekte Darstellung mit CM

\setkomafont{footnote}{\small}  % Marke und Text einer Fußnote, geht auch noch kleiner
                              % bis 2019 usepackage{scrpage2} 
\usepackage{scrlayer-scrpage} % Package laden  2019-09-22 

\usepackage[T1]{fontenc}      % T1-encoded fonts: auch Wörter mit Umlauten trennen
\usepackage[T1]{url}          % much like \verb allow line breaks for paths and URLs
\usepackage[utf8]{inputenc}   % Eingabe nach UTF-8  2016-09-16 

\usepackage{xspace}           % PS Bilder
\usepackage{graphicx}    
\usepackage{graphicx,color}   % JPEG und PNG 
                              % http://en.wikibooks.org/wiki/LaTeX/Colors 
                              
\usepackage{pdfpages}
                               
\usepackage{caption}	      % für subcaption, muss vorher geladen werden 2020-08-03 
\usepackage{subcaption}	      % Mehrere Bilder in einem mit ver. Bildunterschriften 2020-08-03 
\usepackage{wrapfig}	      % Text um Bild

\usepackage[ngerman]{babel}   % Neue deutsche Rechtschreibung  120412ar
%\usepackage[babel,german=guillemets]{csquotes} % für quotes
\usepackage[babel,german=quotes]{csquotes} % für quotes

%tools}             % enthält: afterpage, array, bm, calc, dcolumn, delarray, enumerate, fileerr, fontsmpl, ftnright, hhline, indentfirst, layout, longtable, multicol, rawfonts, showkeys, somedefs, tabularx, theorem, trace, varioref, verbatim, xr, and xspace.
\usepackage{float}

\usepackage{latexsym}         % Sonderzeichen  für ??
\usepackage{pifont}           % dito
\usepackage{array}            % für aufwändigere Tabellen
\usepackage{longtable}        % seitenübergreifende Tabellen passt zu KOMA
\usepackage{multicol}         % Mehrspaltiger Satz
\usepackage{paralist}         % zB für compactitem
\usepackage{tabularx}

\usepackage{makeidx}          % für Index-Erstellung 
%\usepackage{listings}         % für Latex Quelltext
%\usepackage{textcomp}         % for upright mu (\textmu)
\usepackage[fleqn]{amsmath}   % um Gleichungen linksbündig ggf. mit Einzug zu formatieren
\usepackage{amssymb}          % Symbole  
\usepackage{eqnarray}	        % nummerierte und unnummerierte Gleichungen/systeme
\usepackage{ifthen}           % logische Abfragen bei der Formatierung
\usepackage{fancybox}         % für schattierte ovale Boxen etc. geht nicht mit Miktex 5 060614ar
\usepackage{lscape}           % für landscape
\usepackage[normalem]{ulem}   % Unterstreichen und durchstreichen  Probleme mit bibtex
\usepackage{ziffer}           % Komma in math. Umgebung ohne folgendes Leerzeichen
\usepackage{cancel}           % Kürzen in Brüchen 
\usepackage{siunitx}          % für Zahlen und Einheitem 2016-03-09  
%\usepackage{wasysym}          % für \Square 

\usepackage{booktabs}         %  toprule midrule 

\usepackage{todonotes}

%% 2016-09-16 
\usepackage[backend=biber, language=ngerman, 
style=numeric, 
%citestyle=authortitle, 	%
%style=ieee,      % geht nicht
%citestyle=ieee, 	%
sorting=nyt,      % QuellenSortierung none  nyt: Name Titel Jahr     nty
giveninits=true,	% Vornamen abkürzen *** geht wenn Norm in Author steht und ~ 2017-07-31 
%firstinits=true,	% Vornamen abkürzen *** geht wenn Norm in Author steht und ~
maxbibnames=99, 	% Alle Autoren (kein et al.)
maxcitenames=1, 	% Kürzel nur aus 1. Autor
backref=true, 		% Rückverweise auf Zitatseiten
maxnames=50,
block=none]%
{biblatex}
%http://www.khirevich.com/latex/biblatex/

\ExecuteBibliographyOptions{%
	bibencoding=utf8, % wenn .bib in utf8
	bibwarn=true,     % Warnung bei fehlerhafter bib-Datei
}%
% style: numeric [1], alphbetic [Knu01], authoryear,
%
\DefineBibliographyStrings{ngerman}{andothers={et\ al\adddot}} % "u.a." zu "et al." 
\DefineBibliographyStrings{ngerman}{and={;}} % "und" zu ";"
\DefineBibliographyStrings{ngerman}{urlseen={online - zuletzt aufgerufen am}}
\DefineBibliographyStrings{ngerman}{bibliography={Literaturverzeichnis}}
\setcounter{biburllcpenalty}{7000}
\setcounter{biburlucpenalty}{8000}

%%\DeclareNameAlias{default}{last-first}  % erst Nachname, Vorname: dann wird Normeintrag zerruppt
\DeclareNameAlias{default}{family-given}  % 2019-09-23 

\usepackage[plainpages=false,pdfpagelabels]{hyperref}         
%
%% **** END OF CLASS MStyle ****

%schwierigkeit  numerisch nach normzahlen klappt nicht

%\usepackage[
%style=authoryear-icomp,    % Zitierstil
%isbn=false,                % ISBN nicht anzeigen, gleiches geht mit nahezu allen anderen Feldern
%pagetracker=true,          % ebd. bei wiederholten Angaben (false=ausgeschaltet, page=Seite, spread=Doppelseite, true=automatisch)
%maxbibnames=50,            % maximale Namen, die im Literaturverzeichnis angezeigt werden (ich wollte alle)
%maxcitenames=3,            % maximale Namen, die im Text angezeigt werden, ab 4 wird u.a. nach den ersten Autor angezeigt
%autocite=inline,           % regelt Aussehen für \autocite (inline=\parancite)
%block=space,               % kleiner horizontaler Platz zwischen den Feldern
%backref=true,              % Seiten anzeigen, auf denen die Referenz vorkommt
%backrefstyle=three+,       % fasst Seiten zusammen, z.B. S. 2f, 6ff, 7-10
%date=short,                % Datumsformat
%]{biblatex}
%\setlength{\bibitemsep}{1em}     % Abstand zwischen den Literaturangaben
%\setlength{\bibhang}{2em}        % Einzug nach jeweils erster Zeile



%\ExecuteBibliographyOptions{firstinits=true}% Vornamen abkürzne
%\ExecuteBibliographyOptions{%
%maxbibnames=99, % Alle Autoren (kein et al.)
%maxcitenames=1, % Kürzel nur aus 1. Autor
%backref=true, % Rückverweise auf Zitatseiten
%}%
