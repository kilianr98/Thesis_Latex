\chapter{Einleitung}
\label{cha:Einleitung}

\iffalse

%Behauptung
Die kontinuierliche Weiterentwicklung der Gebäudeautomatisierung und die zunehmende Integration von Smart-Home-Systemen haben die Nachfrage nach effizienten und vielseitigen Schalteinrichtungen für die Hausinstallationstechnik erheblich gesteigert. Elektronische Schalter spielen dabei eine Schlüsselrolle, da sie nicht nur die Fernsteuerung elektrischer Lasten ermöglichen, sondern auch dazu beitragen, den Energieverbrauch zu optimieren und die Umweltbelastung zu reduzieren.
Vor diesem Hintergrund ist es von entscheidender Bedeutung, die verschiedenen Ansätze zur Schaltung von \SI{230}{\volt}-Lasten in Hausinstallationssystemen eingehend zu untersuchen.

%Relevanz
Der Bedarf an einer solchen Untersuchung wird durch zwei Hauptfaktoren unterstrichen:
\begin{enumerate}
	\item Die wachsende Nachfrage nach automatisierten Haustechniklösungen, die eine präzise und energieeffiziente Verwaltung elektrischer Lasten ermöglichen.
	\item Die wachsende Bedeutung von Energieeffizienz und Nachhaltigkeit, die den Einsatz effizienter Schaltlösungen unerlässlich macht.
\end{enumerate}

%Thema vorstellen
Diese Bachelorarbeit widmet sich daher einer umfassenden Analyse und dem Vergleich von von elektronischen Schaltertypen, die speziell für die Unterputzmontage in der Hausinstallationstechnik entwickelt wurden. Die betrachteten Schalter umfassen bistabile Relais, MOSFET und Triac.
Dabei wird ein besonderes Augenmerk auf die spezifischen Eigenschaften und Leistungsparameter gelegt, die für die Unterputzmontage von Relevanz sind. Die leitungsgebundenen Störungen, die bei der Verwendung dieser Schalteinrichtungen auftreten können, werden ebenfalls untersucht, um ein umfassendes Verständnis für die praktische Anwendung und die damit verbundenen Herausforderungen zu schaffen.
Zudem werden die verwendeten Schaltungen auf ihre Funktionsweise und verwendeten Schutzmaßnahmen untersucht.

%Ziel vorstellen
Das Ziel dieser Arbeit ist es, eine fundierte Entscheidungsgrundlage für Planer, Entwickler und Endverbraucher zu schaffen, die vor der Wahl der passenden Schalteinrichtung für ihre spezifischen Bedürfnisse stehen. Abschließend werden die gewonnenen Erkenntnisse im Fazit zusammengefasst und ein Ausblick auf zukünftige Entwicklungen und Forschungsfelder im Bereich der Schalteinrichtungen für Unterputzmontage gegeben.

Die Funktionsweise und Bedienung eines Systems mit elektronischen Schaltern wird in \RefFig{fig:System} veranschaulicht. In dieser Darstellung sind die elektronischen Schalteinrichtungen als Elektronik-Einsätze gekennzeichnet, die für das Schalten der angeschlossenen Lasten zuständig sind.
Die Aufsätze auf dem Steckermodul der elektronischen Schalter übertragen die Befehle zum Schalten der angeschlossenen Lasten.
Diese Aufsätze können mit KNX-Funktionalität ausgestattet sein, einem intelligenten Bussystem für die Gebäudesteuerung. Dadurch wird es möglich, Lasten auch dezentral zu steuern und zu schalten, was die Flexibilität und Vielseitigkeit des Systems deutlich erhöht.

\fi