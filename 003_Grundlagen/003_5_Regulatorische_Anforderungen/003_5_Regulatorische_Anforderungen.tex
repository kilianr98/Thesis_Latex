\section{Regulatorische Anforderungen an Feldgeräte}
\label{Regularien}

% Einleitung
Mit der zunehmenden Vernetzung industrieller Systeme gewinnen regulatorische Anforderungen an die Cybersicherheit von Feldgeräten zunehmend an Bedeutung. Neben technischen Schutzmaßnahmen auf Systemebene werden auch konkrete Vorgaben an die sichere Entwicklung, Integration und den Betrieb einzelner Komponenten gestellt. Insbesondere Hersteller von Feldgeräten sind verpflichtet, Security-Aspekte bereits im Entwicklungsprozess zu berücksichtigen und geeignete Schutzmechanismen umzusetzen.

Im Folgenden werden die für Feldgeräte besonders relevanten Anforderungen der IEC~62443-4-2 sowie die regulatorischen Vorgaben des Cyber Resilience Act näher betrachtet.

%IEC 62443-4-2
\subsection{IEC 62443-4-2}
Die Normenreihe IEC 62443 stellt  Anforderungen zur Herstellung von IT-Sicherheit für industrielle Automatisierungs- und Kontrollsysteme  (IACS: Industrial automation and control systems).
Sie umfasst funktionale Anforderungen an  Automatisierungslösungen, -systeme und -komponenten sowie prozessorientierte Vorgehensmodelle für  den Betrieb, die Systemintegration und die Produktentwicklung. Die Norm richtet sich an Hersteller,  Integratoren, Betreiber und besteht aus mehreren Teilnormen \cite{bsi_-_bundesamt_fur_sicherheit_in_der_informationstechnik_ics_2024}.

Für die Entwicklung von Feldgeräten ist insbesondere die Teilnorm IEC~62443-4-2 von Bedeutung.
Sie definiert technische Sicherheitsanforderungen auf Komponentenebene und legt fest, welche Security-Funktionen industrielle Geräte erfüllen müssen, um einem bestimmten Security-Level zu entsprechen.

% Inhalt IEC 62443-4-2
Die IEC 62443-4-2 legt technische Sicherheitsanforderungen für Komponenten industrieller Automatisierungs- und Kontrollsysteme fest.
Grundlage bilden sieben sogenannte grundlegende Sicherheitsanforderungen (Foundational Requirements, FR).
Diese adressieren die Bereiche:
\begin{enumerate}
\item Identifizierung und Authentifizierung,
\item Nutzungskontrolle
\item Systemintegrität
\item Vertraulichkeit von Daten
\item eingeschränkter Datenfluss
\item rechtzeitige Reaktion auf sicherheitsrelevante Ereignisse
\item Verfügbarkeit von Ressourcen
\end{enumerate}


Auf Basis dieser sieben Anforderungen werden Security Levels (SL) definiert, die das angestrebte Schutzniveau gegenüber unterschiedlich leistungsfähigen Angreifern beschreiben.
Jede Anforderung ist in vier Security Levels unterteilt.
Die Level steigen zusammen mit den angenommen Fähigkeiten, Ressourcen und Motivation des potentiellen Angreifers.

Für einzelne Komponenten wird der erreichte Schutzgrad für jede Sicherheitsanforderung separat ausgewiesen. Dies erfolgt über die Bezeichnung SL-C(FR, Komponente) mit einem Wert von 0 bis 4. Ein Level 0 bedeutet, dass für die betreffende Anforderung keine spezifischen Sicherheitsmechanismen gefordert sind, während die Stufen 1 bis 4 steigende technische Schutzmaßnahmen voraussetzen. Die in der Norm beschriebenen Anforderungen werden jeweils einem entsprechenden Security Level zugeordnet.

Kann eine Anforderung nicht allein durch die Komponente erfüllt werden, sind ergänzende Maßnahmen auf Systemebene erforderlich. In solchen Fällen muss der Hersteller geeignete Kompensationsmaßnahmen dokumentieren, die bei der Integration der Komponente in das Gesamtsystem umgesetzt werden müssen \cite{noauthor_iec_2019}.

\subsection{Cyber Resiliance Act}
%Einleitung CRA
Der Cyber Resilience Act (CRA) [4] zielt auf die Verbesse rung der Cybersicherheit für Produkte mit digitalen Elementen ab. Sie legt gemeinsame Cyber-Security-Standards in der gesamten EU fest und verpflichtet die Hersteller, die Security ihrer Produkte während ihres gesamten Lebenszyklus zu gewährleisten \cite{niemann_profinet_2025}.

%was bedeuetet das für die Entwicklung von Feldgeräten
Für die Entwicklung von Feldgeräten bedeutet der CRA vor allem, dass Sicherheitsanforderungen nicht mehr nur „Best Practice“ sind, sondern konformitätsrelevant werden.
Hersteller müssen Bedrohungen und Risiken systematisch bewerten und daraus geeignete technische und organisatorische Maßnahmen ableiten (z. B. Schutz vor unbefugtem Zugriff, Integrität von Firmware/Software, sichere Updatefähigkeit, Vulnerability-Handling).
Orientierung für eine pragmatische Umsetzung bieten u. a. Mappings von CRA-Anforderungen auf bestehende Standards, die zeigen, wie sich CRA-Pflichten mit etablierten Sicherheitsnormen und Entwicklungsprozessen verknüpfen lassen\cite{european_commission_joint_research_centre_cyber_2024}.
So gibt es viele Überschneidungen mit der IEC62443, sodass Unternehmen die Produkte nach dieser Norm bzw~. ihrer Teilnormen entwickeln, somit auch CRA konform sind.

%Was bedeutet das für Geräte die explizit nicht IEC62443-4-2 konform sind?
- Geräte mit z.B. Hart oder anderen nicht ethernet basierten Feldbussen, besitzen oft nur grundlegende bis gar keine Security Maßnahmen
- Laut Definition, sind es aber auch Geräte die kommunizieren und somit unter den CRA fallen
- Da der CRA erst ab dem 11. Dezember 2027 gilt, gibt es noch keine Erfahrung, wie die Handhabung ist
- 

% Aspekt Thesis
Somit ist die Untersuchung dieser Thesis auch im Hinblick des CRA interessant, da aufgezeigt wird, wie auch nicht ethernet basierte Feldgeräte kryptographische Security-Maßnahmen umsetzen können.