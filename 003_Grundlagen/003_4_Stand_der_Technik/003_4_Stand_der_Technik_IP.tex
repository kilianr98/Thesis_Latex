\subsection{Stand der Technik bei netzwerkfähigen Feldgeräten}
\label{2_4_2_Stand_Technik_Ethernet}
Wie in Abschnitt~\ref{Zentrale_Schutzziele} abgegrenzt, fokussiert diese Arbeit auf Anforderungen der Kommunikationssicherheit (Klasse~2), also auf kryptografische Mechanismen für Authentisierung sowie Integritäts- und optional Vertraulichkeitsschutz.
Bei ethernetbasierten Feldgeräten sind diese Mechanismen grundsätzlich verfügbar, da sich die Kommunikation entweder direkt durch Security-Erweiterungen der Feldbusprotokolle absichern lässt oder über etablierte Sicherheitsprotokolle wie TLS abgebildet werden kann.
Der Stand der Technik ist damit nicht durch das Fehlen geeigneter Konzepte geprägt, sondern durch deren praktische Anwendung im Feld.

%\paragraph{Ethernet bis in die Feldebene}
Die zunehmende Vernetzung der OT führt dazu, dass Ethernet-Technologien immer weiter in Richtung Feldgeräte verschoben werden.
Single Pair Ethernet (SPE) bildet hierfür die physikalische Grundlage auf nur einem Adernpaar.
Ethernet-APL ist darauf aufbauend eine prozessindustrielle Ausprägung, die zusätzliche Randbedingungen adressiert, insbesondere die Zweidrahtanbindung mit Energieversorgung, lange Leitungslängen und Konzepte für den Einsatz in explosionsgefährdeten Bereichen.
Damit entsteht technisch die Möglichkeit, ethernetbasierte Kommunikationsprotokolle inklusive ihrer Security-Mechanismen bis zum Messumformer zu führen \cite{ethernet-apl_ethernet_2021}.

%\paragraph{Klasse-2-Mechanismen als etabliertes Muster}
Unabhängig von der konkreten Protokollfamilie folgt Kommunikationssicherheit im Ethernet-Umfeld typischerweise einem wiederkehrenden Muster.
Zunächst werden die Kommunikationspartner beim Verbindungsaufbau authentisiert, häufig auf Basis von X.509-Zertifikaten und einer Vertrauenskette.
Anschließend werden symmetrische Sitzungsschlüssel abgeleitet, da diese für die laufende Datenübertragung deutlich effizienter sind als asymmetrische Verfahren.
Auf dieser Grundlage werden Nachrichten gegen Manipulation geschützt (Integrität und Authentizität) und optional verschlüsselt (Vertraulichkeit).
Die Schlüssellebensdauer wird durch Rekeying oder erneuten Verbindungsaufbau begrenzt \cite{niemann_profinet_2025}.
Genau dieses Muster ist für die spätere Übertragung auf nicht IP-basierte Kanäle relevant.

%\paragraph{Beispiel PROFINET Security}
PROFINET ist ein geeignetes Beispiel, um die Umsetzung von Klasse~2 in einem etablierten Feldbuskontext zu zeigen.
Im Rahmen von PROFINET Security werden Security Classes definiert, die schrittweise Fähigkeiten von Robustheit bis zu kryptografisch geschützter Kommunikation abdecken.
Security Class~1 adressiert vor allem Härtung und Robustness-Aspekte, etwa durch verbesserte Management- und Discovery-Mechanismen sowie die Möglichkeit, Gerätebeschreibungsdateien kryptografisch abzusichern, um Manipulationen in Engineering-Prozessen zu erschweren.
Security Class~2 zielt auf Integrität und Authentizität der Kommunikation, sodass unbemerkte Manipulationen der PROFINET-IO-Daten verhindert werden sollen.
Security Class~3 ergänzt zusätzlich den Vertraulichkeitsschutz, um ein Mitlesen und Interpretieren der Daten zu erschweren, sofern dies im jeweiligen Anwendungsfall erforderlich ist.
Beim Verbindungsaufbau authentisieren sich beide Endpunkte gegenseitig über X.509-Zertifikate, und es werden symmetrische Schlüssel für die nachfolgende Kommunikation abgeleitet.
Diese Schlüssel werden im Betrieb regelmäßig erneuert, um die Auswirkungen einer möglichen Schlüsselkompromittierung zeitlich zu begrenzen.
Damit zeigt PROFINET Security exemplarisch, wie Klasse~2 Anforderungen direkt auf Protokollebene adressiert werden können, ohne dass die Wirksamkeit allein auf Segmentierung oder physische Maßnahmen ausgelagert wird \cite{walz_profinet_2023}.

%\paragraph{Fazit und Bezug zur Arbeit}
Ethernetbasierte Feldgeräte zeigen, dass kryptografisch abgesicherte Verbindungen auf Kommunikationsschnittstellen heute grundsätzlich realisierbar sind und in Spezifikationen bereits vorgesehen werden.
Die wesentlichen Bausteine sind dabei Authentisierung über Geräteidentitäten, Ableitung symmetrischer Sitzungsschlüssel und deren Nutzung für Integritäts- und optional Vertraulichkeitsschutz.
Nicht netzwerkfähige Feldgeräte besitzen diese Bausteine auf ihrer Kommunikationsschnittstelle typischerweise nicht.
Die nachfolgenden Kapitel greifen daher das etablierte Muster aus dem Ethernet-Umfeld auf und übertragen es auf nicht IP-basierte Kanäle, angepasst an deren Ressourcen- und Betriebsrandbedingungen.
