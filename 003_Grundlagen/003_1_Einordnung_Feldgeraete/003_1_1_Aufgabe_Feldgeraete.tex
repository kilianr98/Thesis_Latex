\subsection{Funktion und Aufgaben von Feldgeräten}
\label{003_1_1_Aufgabe_Feldgeraete}
%
%%% Was will ich sagen:
% - Überblick was Feldgeräte sind
% - Was sind Anwedungsgebiete
% - Einfach eine kurze Einleitun
% - GEnauere Einordnung kommt in späteren Kapitel
% - Feldgeräte kommen mit und ohne Display (Also für menschen bzw. Maschinen)
% - Statistik wie viele Feldgeräte weltweit eingesetzt werden (Zahlen aus 2023,24 oder 25)

Feldgeräte nehmen eine zentrale Rolle in industriellen Automatisierungs- und Steuerungssystemen ein. Sie bilden die Schnittstelle zwischen der physischen Welt und übergeordneten Steuerungssystemen, indem sie Daten erfassen, verarbeiten und weiterleiten oder direkt in Prozesse eingreifen.
Zu den typischen Feldgeräten gehören Sensoren, die physikalische Größen wie Temperatur, Druck, Messwerte, Füllstand oder Durchfluss messen, sowie Aktoren, die mechanische Bewegungen oder andere Aktionen ausführen.
Im Fokus dieser Thesis stehen Sensoren, während Aktoren nicht Gegenstand der Untersuchung sind.

Die Einsatzgebiete von Feldgeräten sind äußerst vielfältig und erstrecken sich über nahezu alle Industriezweige.
In der Prozessindustrie, beispielsweise in der Chemie- oder Öl- und Gasindustrie, überwachen sie kritische Parameter, um die Sicherheit und Effizienz von Anlagen sicherzustellen.
In der Fertigungsindustrie ermöglichen Feldgeräte eine präzise Erfassung von Zuständen und Prozessgrößen und bilden die Grundlage für automatisierte Produktionsabläufe.
Auch in der Energieversorgung, etwa in Kraftwerken, Stromnetzen oder der Wasserwirtschaft, sind Feldgeräte unverzichtbar für die Überwachung und Steuerung technischer Anlagen.
Die hier beschriebenen Einsatzmöglichkeiten beziehen sich sowohl auf Sensoren als auch auf Aktoren, die jeweils spezifische Aufgaben in den Prozessen übernehmen.

Feldgeräte unterscheiden sich zudem hinsichtlich ihrer Interaktion mit Menschen und Maschinen.
Während einige Geräte über lokale Anzeige- und Bedienelemente verfügen und eine direkte Bedienung vor Ort erlauben, werden andere Feldgeräte ausschließlich maschinell über Steuerungen, Asset-Management-Systeme oder mobile Servicegeräte angesprochen.

\todo{Eine Statistik wie viele Feldgeräte es weltweit gibt -> VEGA?}

Da Feldgeräte den realen physikalischen Zustand eines Prozesses erfassen und Prozessentscheidungen auf diesen Messwerten basieren, ist ihre zuverlässige und korrekte Funktion von entscheidender Bedeutung.
Fehlerhafte oder manipulierte Messwerte können unmittelbare Auswirkungen auf die Verfügbarkeit, Produktqualität und Sicherheit industrieller Systeme haben.