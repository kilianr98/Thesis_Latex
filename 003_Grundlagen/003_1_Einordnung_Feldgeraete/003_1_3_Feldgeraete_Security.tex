\subsection{Security-Relevante Bedeutung von Feldgeräten}
\label{Feldgeraete_Security}

\subsubsection{Feldgeräte als Einfallspunkt für Angriffe}
%Einleitung
%%Pass
Jedes Feldgerät, das in ein OT-Netzwerk bzw. ein Industrial Control System (ICS) integriert wird, erweitert die Funktionalität des Gesamtsystems und zugleich auch dessen Angriffsfläche.
Abhängig von Fähigkeiten und Kommunikationsschnittstellen, sowie der Einbindung in die Systemarchitektur, können von einzelnen Feldgeräten verschiedene Risiken ausgehen.

% - Wie kann ein Angriff auf Feldgeräte die OT gefährden MC3
Betrachtet man die grundlegende Funktionalität von Feldgeräten, insbesondere von Sensoren an, so lassen sich in den meisten Fällen (rein analoge Geräte ohne zusätzliche Kommunikationsschnittstellen ausgenommen) zwei wesentliche Kommunikationspfade unterscheiden:
Der Sensor-Kanal zur Übertragung von Messwerten an übergeordnete Steuerungen sowie der Control-Kanal, über den Parametrierung, Konfiguration oder Diagnose erfolgt.

% 	- Control Channel MC3, MC5
% 		- Wie läuft das ab MC3
Ein Angriff über den Control-Kanal zielt darauf ab, eine Systemkomponente aus einer höheren Kommunikationsschicht (verweis purdue Modell), zu kompromittieren, um anschließend manipulierte Befehle in das System einzuschleusen\cite{mclaughlin_cybersecurity_2016}.

% - (HART as an attack vector)
In der Praxis kann dies beispielsweise über missbrauchte Feldbus- oder Serviceprotokolle erfolgen. So wurde in \cite{alexander_hart_2014} gezeigt, dass über manipulierte HART-Kommandos nicht nur Feldgeräte beeinflusst, sondern unter bestimmten Voraussetzungen auch weiterführende IT-Systeme, bis hin zur Unternehmensebene, erreicht werden können.
Der Control-Kanal eines Feldgeräts kann somit als Einstiegspunkt dienen, um über legitime Kommunikationsbeziehungen weiter in den OT- oder sogar IT-Bereich vorzudringen.

% 	- Sensor Channel MC4
Im Gegensatz dazu zielen Sensor-Channel-Angriffe auf die Manipulation der vom physikalischen Prozess gelieferten Messwerte.
Hierbei werden Sensordaten verfälscht, sodass Steuerungen oder Leitsysteme auf Grundlage falscher Informationen Entscheidungen treffen.
Ziel ist es, das Verhalten des Reglers gezielt zu beeinflussen oder einen realen Prozesszustand zu verschleiern.
Diese als False-Data-Injection (FDI) bezeichneten Angriffe wurden ursprünglich im Kontext von Energieversorgungssystemen und Smart Grids beschrieben, gelten jedoch aufgrund der zunehmenden Vernetzung industrieller Anlagen als generisches Risiko für ICS-Umgebungen. Da industrielle Prozesse häufig sicherheitskritisch sind und erhebliche ökologische, wirtschaftliche oder gesellschaftliche Auswirkungen haben können, werden Manipulationen von Sensordaten als besonders schwerwiegender Angriffsvektor betrachtet.
So kann beispielsweise eine künstlich abgesenkte Temperaturmessung dazu führen, dass die Heizleistung erhöht wird, obwohl keine tatsächliche Abweichung vorliegt, was im Extremfall zu einer unentdeckten Überhitzung führen kann. \cite{elnour_machine_2023,mclaughlin_cybersecurity_2016}.

\subsubsection{Abgrenzung Safety - Security}
% Zusammenhang und Abgrenzung zu Safety
Cybersicherheit (Security) dient dem Schutz von OT-Systemen vor mutwilligen Manipulationen, die deren bestimmungsgemäßen Betrieb beeinträchtigen oder verhindern können.
Ziel ist es, die Integrität, Verfügbarkeit und Vertraulichkeit der Systeme sowie deren sichere Funktionsfähigkeit aufrechtzuerhalten.
Hierzu zählt insbesondere auch der Schutz sicherheitskritischer Funktionen, die im Rahmen der Funktionalen Sicherheit implementiert sind.

Die Funktionale Sicherheit (Safety) verfolgt das Ziel, Menschen, Umwelt und Anlagen vor Gefährdungen zu schützen, die aus Fehlfunktionen technischer Systeme resultieren können \cite{bsi_-_bundesamt_fur_sicherheit_in_der_informationstechnik_ics_2024}.
Sie adressiert somit unbeabsichtigte Fehlerzustände, während Security vorsätzliche Angriffe berücksichtigt.

Cyberangriffe können jedoch unmittelbar Einfluss auf die Funktionale Sicherheit nehmen, indem sie sicherheitsgerichtete Systeme manipulieren oder außer Kraft setzen.
Ein prägnantes Beispiel hierfür ist die im Jahr 2017 entdeckte TRITON-Malware. Diese zielte auf das Safety Instrumented System (SIS) einer petrochemischen Anlage in Saudi-Arabien ab und versuchte, dessen Schutzfunktionen gezielt zu manipulieren.
Dadurch wurde die Fähigkeit des Systems, gefährliche Prozesszustände zu erkennen und abzusichern, beeinträchtigt, was potenziell zu schweren Personen- und Umweltschäden hätte führen können \cite{di_pinto_triton_2018}.
Der Vorfall verdeutlicht, dass Security-Schwachstellen direkte Auswirkungen auf die Safety eines Systems haben können.

Obwohl Safety und Security unterschiedliche Zielrichtungen verfolgen und jeweils eigene normative Rahmenwerke besitzen, sind sie in OT-Umgebungen eng miteinander verknüpft. Während Safety den Schutz von Menschen, Umwelt und Anlagen durch das System adressiert, zielt Security auf den Schutz des Systems vor externer Manipulation ab \cite{bsi_-_bundesamt_fur_sicherheit_in_der_informationstechnik_ics_2024}.
Im deutschen Sprachgebrauch wird der Begriff „Sicherheit“ häufig für beide Aspekte verwendet. Sofern in dieser Arbeit nicht ausdrücklich anders gekennzeichnet, bezieht sich der Begriff auf Security im Sinne der Informations- und Cybersicherheit.