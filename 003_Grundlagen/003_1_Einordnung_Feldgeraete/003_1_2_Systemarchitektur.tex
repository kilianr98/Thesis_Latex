\subsection{Systemarchitekturen und Einbindung von Feldgeräten}
\label{2_1_2_Systemarchitektur_Feldgeraete}

%% Was will ich hier sagen:
% - Kurzer Absatz zu ICS/ OT wie generell so was organisiert ist
% - Erklären was die Automatisierungspyramide ist
% -- Schichten grob erklären
% - Begriffserklärung bzw. auch entscheiden ob ich von OT oder ICS spreche
% - Feldgeräte in die Automatisierungspyramide einordnen
% - 


%% Erklärung Was die ~~Automatisierungspyramide~~ Purdue Modell ist (Abgrenzung zu automatisierungspyramide)
% PASS
Zur Einordnung von Funktionen, Systemen und Kommunikationsbeziehungen in industriellen Umgebungen wird häufig das Purdue-Modell (auch als Purdue Enterprise Reference Architecture, PERA, referenziert) verwendet.
Es beschreibt ein hierarchisches Ebenenkonzept für industrielle Produktions- bzw.\ Prozesssysteme und strukturiert die Aufgabenverteilung von der operativen Prozessausführung bis zur unternehmensweiten Planung.
Dabei wird zwischen horizontaler Kommunikation (innerhalb einer Ebene) und vertikaler Kommunikation (zwischen unterschiedlichen Ebenen) unterschieden. Für die Ebenen 0 bis 4 ist das Modell weitgehend kompatibel mit dem in der Praxis verbreiteten fünfstufigen Ebenenkonzept der Automatisierungspyramide.
Im Purdue-Ansatz werden jedoch zusätzlich Zonen zur Abgrenzung und Kopplung unterschiedlicher Domänen berücksichtigt, insbesondere eine Übergangszone (Level 3.5, OT-DMZ) sowie eine externe bzw.\ Internet-nahe Zone \cite{babel_systemintegration_2024}.
Damit rückt weniger die reine funktionale Hierarchie als vielmehr die Netzsegmentierung und die kontrollierte Gestaltung von Übergängen in den Vordergrund, um Kommunikationsflüsse zwischen Office-IT, OT/ICS und externen Netzen gezielt zu steuern und abzusichern \cite{bsi_-_bundesamt_fur_sicherheit_in_der_informationstechnik_ics_2024}.

%%Beschreibung der Abbildung
In \RefFig{fig:Purdue} ist das Purdue-Modell als hierarchische Referenzarchitektur für industrielle OT/ICS-Umgebungen dargestellt.
Die Abbildung verdeutlicht die Anordnung der Ebenen sowie deren typische Kopplungspunkte und Schnittstellen. Darüber hinaus sind beispielhafte Kommunikationspfade zwischen den Ebenen eingezeichnet, wodurch sowohl horizontale Informationsflüsse innerhalb einer Ebene als auch vertikale Informationsflüsse zwischen den Ebenen nachvollziehbar werden.
Ergänzend zeigt die Darstellung den Einsatz von Sicherheitskomponenten wie Firewalls und unidirektionalen Übertragungseinrichtungen (Datendioden), mit denen Kommunikationsbeziehungen segmentiert und Datenflüsse gezielt auf eine Richtung beschränkt werden können.

\begin{figure}[h]
	\begin{center}
		\includegraphics[width=1\textwidth]
		{003_Grundlagen/.assets/Purdue.png}
		\caption[Beschreibung des Schalter-Systems]
		{\label{fig:Purdue}
			Beispiel Netzwerk nach Purdue/IEC 62443
			Bildquelle: \autocite{deutschland_it-grundschutz-kompendium_2023}}
	\end{center}
\end{figure}

%Einordnung der Ebenen
\subsubsection{Einordnung in Ebenen des Purdue-Modells}
\label{2_1_2_Einordnung_in_Ebenen_des_Purdue_Modells}
% Ebene 5
%Pass
Das Purdue-Modell ergänzt oberhalb der Produktionsführungs- und Unternehmensebene noch eine Internet Ebene, Ebene~5, welche die typische Kommunikation mit dem Internet (Web, Mail) repräsentiert.

% Ebene 4
%Pass
Auf Ebene~4 (Unternehmensebene) findet typischerweise unter Nutzung eines ERP-Systems die übergeordnete Planung und Koordination betriebswirtschaftlicher Abläufe statt.
Dazu zählen insbesondere die Grobplanung der Produktion sowie unterstützende Funktionen für Organisationsbereiche wie Vertrieb (z.\,B. Erfassung von Kundenaufträgen) und Einkauf (z.\,B. Beschaffung von Materialien), welche in einem ERP-System abgebildet werden können \cite{babel_systemintegration_2024}.

%Ebene 3.5 
%Pass
Eine weitere, wichtige Erweiterung ist die Übergangszone Ebene 3.5 (OT-DMZ) zwischen der Office-IT und der Produktion.
Als Demilitarized Zone verhindert diese Zone eine direkte Kommunikation zwischen den beiden Segmenten.
Informationen werden ausschließlich über in der DMZ bereitgestellte Schnittstellen ausgetauscht.
Idealerweise wird die Verbindung hierbei von der Zone mit dem höheren Schutzbedarf aus aufgebaut. Da das ICS (Industrial Control System) in der Regel einen höheren Schutzbedarf als die Office-IT aufweist, wird die Verbindung von dieser Seite initiiert.
So dürfen zum Beispiel ICS-Systeme Daten auf eine Datenbank in der DMZ schreiben, die Office Systeme hier aber nur lesend zugreifen.

% Ebene 3
% Pass
Auf Ebene~3 (Betriebsleitungsebene) erfolgt eine detailliertere Planung und Steuerung der Produktion.
Hier kommen häufig Manufacturing Execution Systems (MES) an den jeweiligen Produktionsstandorten zum Einsatz.
Ein MES-System überwacht, steuert und optimiert in Echtzeit alle produktionsnahen Prozesse, einschließlich Betriebs-, Maschinen- und Personaldatenerfassung, sowie Material-, Qualitäts- und Energiemanagement, um eine effiziente Fertigung sicherzustellen \cite{babel_systemintegration_2024}.
Diese Ebene bildet die Schnittstelle zwischen der betriebswirtschaftlich orientierten Organisationsebene und den operativen Produktions- und Automatisierungssystemen.

% Ebene 2
%Pass
Die Überwachung und operative Prozessführung erfolgt auf Ebene~2 (Prozessleitungsebene).
Auf dieser Ebene werden typischerweise Supervisory Control and Data Acquisition (SCADA)-Systeme sowie Prozessleitsysteme (PLS) zur Produktionsdatenerfassung, -visualisierung und -kontrolle eingesetzt.
Sie unterstützen unter anderem die Anzeige und Auswertung von Betriebsdaten sowie die Überwachung von Anlagenzuständen und Prozessparametern.\cite{babel_systemintegration_2024}.

% Ebene 1
%Pass
Auf Ebene~1 (Prozesssteuerungsebene) übernehmen speicherprogrammierbare Steuerungen (SPS; engl.\ PLC) und zugehörige Ein-/Ausgabekomponenten (I/O) die lokale Steuerung und Regelung.
Über diese Komponenten werden Signale aus der Feldebene verarbeitet und Stellgrößen an den Prozess ausgegeben.
Die Steuerungsebene wirkt damit unmittelbar auf den Prozess ein.

%% Einordnen der Feldgeräte (Ebene 0)
%Pass
In der Feldebene (Ebene~0) befinden sich die Komponenten, die Informationen aus dem materiellen Produktions- bzw. Prozessgeschehen erfassen oder als Aktoren direkt darauf einwirken.
Dazu zählen beispielsweise Endschalter und Sensoren, die im Folgenden als Feldgeräte zusammengefasst werden.
Diese Komponenten interagieren einerseits direkt mit dem physikalischen Prozess und andererseits, über eine zugehörige Infrastruktur (z.\,B. Anschluss- und Kopplungskomponenten), mit den informationsverarbeitenden Einheiten der darüberliegenden Ebenen.
Für die Kommunikation auf Ebene~0 besteht grundsätzlich die Notwendigkeit, Sensordaten und Aktorbefehle unter deterministischen bzw. echtzeitnahen Bedingungen zu übertragen. Zusätzlich müssen bei Bedarf Diagnose- und Konfigurationsdaten übermittelt werden, etwa für Inbetriebnahme, Wartung oder Parametrierung \cite{bsi_-_bundesamt_fur_sicherheit_in_der_informationstechnik_ics_2024}.


%% Kommikation der Schichten
\subsubsection{Kommunikation der Schichten}
\label{2_1_2_Kommunikation_der_Schichten}
%Pass
Die horizontale und vertikale Kommunikation wird in der Praxis häufig über Feldbus- und Automatisierungsnetzwerke realisiert, die je nach Systemarchitektur und Generation sowohl ethernetbasiert als auch nicht ethernetbasiert ausgeprägt sein können.

Die Kommunikation in ICS ist nicht auf die jeweilige Ebene beschränkt.
So kann der Wert eines Füllstandsensor eines Ventils auf Ebene~0 über eine SPS auf Ebene~1 an eine Software auf Ebene~2 übertragen werden.
Für die ebenenübergreifende Kommunikation kommen häufig Gateways zum Einsatz.
Das Gateway (Ebene 1) wandelt Daten des I/O-Subsystems auf dem Feldbus (Ebene 0) in ein anderes Protokoll um und leitet diese an ein System auf Ebene 2 weiter.
Von dort wird die Kommunikation zu Ebene 3 und 4 jeweils durch eine Firewall gefiltert und über die DMZ, die als Sicherheitszone eine direkte Kommunikation zwischen Netzwerken verhindert, geleitet.
So können Daten zwischen verschiedenen Systemen ausgetauscht werden, aber nicht jedes System muss mit jedem direkt kommunizieren. Das ERP-System benötigt zum Beispiel keine Sensordaten von I/O Systemen auf dem Feldbus \cite{bsi_-_bundesamt_fur_sicherheit_in_der_informationstechnik_ics_2024}.

% Ethernet basiert nicht ethernet basiert
\todo{Absatz unterschied ethernet basiert und nicht}

%%4..20mA HART
%%Pass
In bestimmten Industriebereichen, insbesondere in der Prozessindustrie, sind zudem weiterhin zahlreiche Feldgeräte im Einsatz, die Messwerte über eine 4--20\,mA Stromschleife analog liefern.
Häufig wird dies durch eine zusätzliche digitale Kommunikation ergänzt die wenig Energie benötigt und über die Konfigurations- oder Diagnosedaten übertragen werden können (z.\,B. über HART) \cite{niemann_ot-sicherheitsanforderungen_2022}.

%% Drathlose Kommunikation
%%Pass
Drahtlose Kommunikation kann ebenfalls Bestandteil horizontaler und vertikaler Kommunikationsstrukturen sein.
Da der Fokus dieser Arbeit jedoch auf kabelgebundenen Kommunikationspfaden liegt, wird drahtlose Kommunikation im weiteren Verlauf nicht vertieft.

%% Unterscheidung netzwerkfähig und nicht netzwerkfähig Feldgeräten bzw. Feldbussen

% Unteschied OT/IT
\subsubsection{Abgrenzung OT/IT}
\label{2_1_2_Abgrenzung_OT_IT}
%Pass
Die in der Vergangenheit übliche physische Trennung der OT von anderen IT-Systemen und Datennetzen in Büroanwendungen ist heute aufgrund zunehmender Integrationsanforderungen nur in Ausnahmefällen bei erhöhtem Schutzbedarf anwendbar.
Mehrstufige Produktionsschritte und deren übergreifende Steuerung sowie regulatorische Anforderungen machen es zunehmend notwendig, die OT auch über Organisationsgrenzen hinweg zu öffnen.
Dieser Prozess wird häufig als IT/OT-Konvergenz bezeichnet, ein Begriff, der die zunehmende Verschmelzung von Informationstechnologie (IT) und Betriebstechnologie (OT) beschreibt.
Diese Entwicklung wird durch den Trend zur Optimierung von Fertigungsprozessen noch beschleunigt, vor allem im Rahmen der Industrie 4.0.\cite{bsi_-_bundesamt_fur_sicherheit_in_der_informationstechnik_ics_2024}.