\section{Zentrale Schutzziele für Feldgeräte}
\label{Zentrale_Schutzziele}

%Einleitung
% - Was beschreiben Schutzziele
Die Sicherheit moderner IT- und OT-Systeme stützt sich unter anderem auf das Konzept der Informationssicherheit.
Dieses umfasst Maßnahmen und Strategien, die darauf abzielen, Systeme, Daten und Kommunikation vor unbefugtem Zugriff, Manipulation und Ausfällen zu schützen
Informationssicherheit bildet eine wesentliche Grundlage für die Entwicklung sicherer Feldgeräte und damit auch für den Aufbau zuverlässiger und sicherer Anlagen.

Ein zentrales Element der Informationssicherheit sind sogenannte Schutzziele.
Diese beschreiben, welche sicherheitsrelevanten Eigenschaften eines Systems oder einer Komponente erhalten bleiben müssen, um einen sicheren Betrieb zu gewährleisten.
Für Feldgeräte, die in sicherheitskritischen Umgebungen eingesetzt werden, sind Schutzziele von besonderer Bedeutung, da sie die Grundlage für den Schutz vor Angriffen und die Gewährleistung eines zuverlässigen Betriebs bilden.

%CIA Triade
\paragraph{Die CIA-Triade und deren Anwendung in OT-Systemen}

Die CIA-Triade ist ein zentrales Konzept der Informationssicherheit und definiert drei grundlegende Schutzziele:

\begin{itemize}
  \item Geheimhaltung (Confidentiality)
  \item Integrität (Integrity)
  \item Verfügbarkeit (Availability)
\end{itemize}

Sie dient als Grundlage für die Bewertung und den Schutz von IT- und OT-Systemen.

Während in IT-Systemen die Geheimhaltung oft oberste Priorität hat, stehen in OT-Systemen die Integrität und Verfügbarkeit im Vordergrund. Dies liegt daran, dass ein Systemausfall oder die Manipulation von Daten direkte Auswirkungen auf physische Prozesse haben kann. Die Vertraulichkeit von Daten spielt hier im Vergleich eine geringere Rolle \cite{stouffer_guide_2023}.

Die Normenreihe IEC 62443-4-2 konkretisiert diese Schutzziele auf Komponentenebene und definiert sieben Foundational Requirements (FR), die als normative Schutzziele interpretiert werden können.
Diese Anforderungen adressieren zentrale Sicherheitsaspekte wie Authentifikation, Zugriffskontrolle und Integrität und bieten einen klaren Rahmen für die Entwicklung sicherer Feldgeräte.
Es wurde auch noch das Schutzziel \emph{Organisation} hinzugefügt, das verdeutlicht, das diese Anforderungen mittels organisatorischer Maßnahmen umgesetzt werden müssen. \todo{Die Zuordnung in Schutzziel macht keinen Sinn}

\todo{Tabelle sauber beschrieben in Latex einfügen.}
%Mapping der Anforderungen
\begin{figure}[h]
	\begin{center}
		\includegraphics[width=1\textwidth]
		{003_Grundlagen/003_3_Zentrale_Schutzziele/Mapping_Schutzziele.png}
		\caption[Mapping der Anforderungen]
		{\label{fig:Mapping_FR}
			Mapping der Anforderungen -> DELETE
        }
	\end{center}
\end{figure}

% Abgrenzung der betrachteten Anforderungen
Da sich diese Thesis mit dem sicheren Verbindungsaufbau bei Feldgeräten ohne Netzwerkschnittstelle befasst, werden Anforderungen, die ausschließlich durch organisatorische oder bauliche Maßnahmen in der Umgebung umgesetzt werden können, im weiteren Verlauf nicht vertieft.
Für die Eingrenzung des Untersuchungsumfangs werden die aus IEC~62443-4-2 abgeleiteten Anforderungen danach unterschieden, ob sie durch den Kommunikationsmechanismus adressierbar sind oder außerhalb des Einflussbereichs eines Protokolls liegen.
Daraus ergibt sich eine Einteilung in drei Klassen, nach \cite{niemann_profinet_2025}:

\begin{enumerate}
  \item Anforderungen, die für das betrachtete Feldgerätprofil nicht relevant sind (z.\,B. Notstromversorgung, Schutz der Zonengrenze).
  \item Anforderungen, die unmittelbar durch kryptografische Mechanismen auf der Kommunikationsstrecke umgesetzt werden können (z.\,B. Integritätsschutz und Vertraulichkeit der übertragenen Daten).
  \item Anforderungen, die sicherheitsrelevant sind, deren Umsetzung jedoch primär von der Geräteplattform abhängt (z.\,B. Integrität beim Software-Update, Integrität des Boot-Prozesses).
\end{enumerate}

Die vorliegende Arbeit fokussiert daher auf Klasse~2, da nur diese Anforderungen direkt durch den Verbindungsaufbau, die Authentisierung und die Aushandlung von Sitzungsschlüsseln auf Protokollebene beeinflussbar sind.
Angriffe auf die Verfügbarkeit, die durch physischen Zugriff oder Zerstörung des Geräts entstehen, können durch ein Kommunikationsprotokoll hingegen nicht verhindert werden.
