\section{Stand der Technik bei nicht netzwerkfähigen Feldgeräten}
\label{Stand_Technik_Non_IP}

% Was ich in dem Kapitel sagen will:
% - Aufgrund mangelender Rechenleistung kann keine 

% Onboarding von Devices, wie wird ein Grät identifiziert?
% - Typenschild?

% Auszug aus BSI, 3.2.1.4 Man in the middle Angriff - unverschlüsselte Kommunikationen
Die Hauptpriorität der Sicherheit von OT wird oftmals in der Verfügbarkeit und Zuverlässigkeit der Systeme gesehen.
Aspekte der Vertraulichkeit und Integrität werden unter Umständen nachrangig berücksichtigt.
Bspw. wird häufig auf eine Verschlüsselung der Daten oder Transportwege verzichtet.
Hieraus entsteht die Gefahr das Daten von Angreifern abgefangen sogar manipuliert werden können.
In einem solchen Fall ist die Integrität und die Vertraulichkeit der Daten nicht mehr sichergestellt. 
Ein Angreifer mit physischem Zugriff auf das OT-Netz kann diese Werte somit auslesen, verändern oder neue einspielen (z. B. zur Steuerung einer Maschine oder zur Fälschung von Sensordaten \cite{bsi_-_bundesamt_fur_sicherheit_in_der_informationstechnik_ics_2024}.

% Warum gibt es keine krypto?


% Wie wird geschützt?
% Air Gapped
Als Schutzmaßnahme gegen Cyberangriffe wurden speziell ältere Anlagen eine Zeit lang physikalisch von anderen Netzen getrennt.
Dies gilt insbesondere für solche, in denen Systeme mit bekannten Schwachstellen enthalten sind oder eine unzureichende Zugangskontrolle bieten.
Diese so genannten „Air-Gaps“ bieten jedoch selten das angestrebte Schutzniveau gegen Cyberangriffe.
Denn in vielen Fällen ist weiterhin ein Datenaustausch notwendig oder erwünscht.
Die hierfür eingesetzten Daten können von Angreifern genutzt werden, um die Trennung zu überwinden.

% Geräte im Geschützten Bereich
Feldgeräte werden oft im geschützten Bereich eingesetzt, dass bedeutet, dass keine unbefugten Personen Zutritt zum Feldgerät haben.
Zum Beispiel durch eine Pforte. Angenommen, diese Maßnahme würde tatsächlich den Zutritt von unbefugten Personen wirksam unterbinden, schützt das nicht
gegen Angreifer von Innen, sprich Personen die Zutritt haben.

Zudem gibt es auch Anwendungsszenarien, in dem ein geschützter Bereich nicht möglich ist, und Feldgeräte für jeden frei zugänglich sind.
Als Beispiel sein hier Stauseen genannt.


% Hardware
Bei vielen Feldgeräten spielt der Energieverbrauch eine sehr große Rolle. Es gibt einen maximalen Wert der nicht überschritten werden kann. Bei 
Berechnungen von Kryptooperationen ohne spezielle Kryptographische Prozessoren sind rechenintensiv. Dadruch, dass harte Grenzen beim Energieverbrauch gelten,
und der Takt vom Mikrocontroller generell gedrosselt ist, können solche Operationen zu langen Rechenzeiten führen, die nicht mehr akzeptabel sind.
(Hier irgendwie auf einen Test oder so verweisen). Aufgrund dessen wird häufig auf Kryptographische Operationen verzichtet, bzw. sind auch aufgrund 
der vorgegebenen Anforderungen an Energieverbrauch schlichtweg nichtmöglich.

Durch den Trend, auch Feldgeräte "smart" miteinander zu vernetzen, steigt auch der Wunsch und Nachfrage auch in ressourcenbeschränkten Geräten Kryptooperationen
durchzuführen.
Somit gibt es viele Mikrocontroller die kryptographische Berechnungen in einem dafür gemachten HW-Bereich durchführen. Damit können kryptographische Operationen schneller und 
energieeffizienter durchfügehrt werden. Dadurch kann der eigentliche Chip immernoch langsam, bzw. Energieeffizient sein, aber gleichzeitig die Anferderung nach
schnellen Kryptographischen Operationen erfüllen.
Somit werden 

Feldgeräte werden über längere Zeiträume mit gleicher Hardware betrieben. Die Lebensdauer beträgt zwischen 10 und 15 Jahren.
Daher sind noch viele Feldgeräte im Einsatz, die noch ältere Leistungs- und Effizienzschwächere Hardware verwenden.
Dadurch dauert es eine lange Zeit, bis sich neuere Trends z.B. die Verwendung von Krypto HW durchsetzt. 


% Was ändert sich gerade?
% - Mehr Kryptoprozessoren

% Thema begrenzter Strom
Bei vielen industriellen Feldgeräten, insbesondere  Feldgeräte mit 2-Draht-Technik ist der verfügbare Energiehaushalt sehr stark begrenzt.
Der Strom für die gesamte Elektronik (Sensor, A/D-Wandlung, Signalverarbeitung, Kommunikation) muss typischerweise aus wenigen Milliampere der Stromschleife bereitgestellt werden.
Designrichtlinien wie \cite{johnson_power_2013} nennen Budgets von 3 - 3,5~\si{\milli\ampere} für die interne Elektronik, die nicht überschritten werden dürfen,
damit der Messbereich von 4 - 20~\si{\milli\ampere} eingehalten wird.

Kryptographische Verfahren, die rein in Software auf einem Mikrocontroller, ohne spezielle Krypto-Peripherien, durchgeführt werden,
sind im Vergleich zu klassischer Signalverarbeitung sehr rechen- und energieintensiv.
Während die meisten Feldgeräte wenig Energie-/ Leistungsreserven besitzen, müssen die Krypto-Operationen auch noch in die Reserven passen.

Ein möglicher Ansatz, um diesen Konflikt zu umgehen, ist der Einsatz von Crypto-Peripherie. Diese lagert die Berechung in dedizierte HW aus, die speziell dafür konstruiert wurde und die Ausführungszeit,
sowie auch den Energieverbauch pro Operation deutlich reduzieren.

Ein weiterer Vorteil ist auch die erhöhte Sicherheit, da weitere Security Mechanismen wie sichere Erzeugung von privaten Schlüsseln, Zertifizierte Entropie, Anti-Tampering-Maßnahmen und weitere Features bietet.

Werden eigene
Messungen auf dem (auch hier in der Arbeit verwendeten Mikrocontroller) zeigen, dass

Ein weiterer Aspekt ist die lange Einsatzdauer industrieller Feldgeräte. Komponenten der Betriebstechnik (OT) werden in industriellen Steuerungssystemen typischerweise über Zeiträume von 10 bis 15 Jahren oder länger betrieben,
deutlich länger als klassische IT-Hardware (Zitat). Das bedeutet, dass heute noch eine große installierte Basis von Feldgeräten mit älterer, nicht kryptofähiger Hardware im Feld ist.
Die Modernisierung hin zu Geräten mit integrierter Krypto-Hardware und damit die breite Umsetzung kryptografisch gesicherter Verbindungen bis hinunter zum Feldgerät erfolgt daher nur schrittweise im Rahmen von Migrations- und Retrofit-Projekten und wird durch Lebensdauer, Zertifizierungen (z. B. ATEX/IECEx) und die hohen Kosten von Gerätewechseln zusätzlich verlangsamt.
