%%%%%%%%%%%%%%%%%%%%%%%%%%%%% Gesetzt %%%%%%%%%%%%%%%%%%%%%%%%%%%%%%%%%%%%%%%%%%%%%%%%%%
% Stand der Technik bei nicht netzwerkfähigen Feldgeräten
\section{Stand der Technik bei nicht netzwerkfähigen Feldgeräten}
\label{Stand_Technik_Non_IP}
\sloppy 


\todo{Beschreibung des Kapitels einfügen}


% ## 1. Einordnunug und Begriffsdefinition
% ### 1.1 Was heißt nicht-netzwerkfähige Geräte
\subsection{Einordnung und Begriffsdefinition}
Im Rahmen dieser Arbeit werden unter nicht-netzwerkfähigen Geräten solche Feldgeräte verstanden, die entweder ausschließlich einen kontinuierlichen Messwert bereitstellen oder zwar mit einer Steuerungs- oder Leitebene kommunizieren, selbst jedoch keinen eigenen Netzwerk- oder IP-Stack implementieren.
\todo{Abschnitt einfügen was Netzwerk-Stack ist}

%%%
Typische Vertreter dieser Geräteklasse sind klassische Prozessfeldgeräte wie Druck-, Temperatur-, Durchfluss- oder Füllstandssensoren sowie Grenz- und Näherungsschalter.
Sie sind üblicherweise über 4-20-mA-Stromschleifen, über HART oder über feldbusbasierte Systeme wie PROFIBUS-PA oder vergleichbare Feldbusse an eine übergeordnete Steuerung angebunden.
In der Automatisierungspyramide sind diese Geräte der Feldebene (Level 0) zuzuordnen, wie in Abbildung \todo{insert ref} dargestellt.
\todo{Was ist die AUtomatisierungspyramide}

\missingfigure{Automatisierungspyramide}

Sie erfassen physikalische Größen direkt im Prozess oder wirken unmittelbar auf diesen ein und bilden damit die Schnittstelle zwischen physikalischer Anlage und digitaler Steuerung.
%%%

%%%
Zu den nicht-netzwerkfähigen Geräten im Sinne dieser Arbeit zählen ebenfalls Feldgeräte, die keine direkte Verbindung zu einer übergeordneten Steuerung besitzen, sondern deren Messwerte ausschließlich lokal bereitgestellt werden, beispielsweise über ein angeschlossenes Anzeige- oder Bediengerät.
In solchen Fällen wird der Messwert ausschließlich von einem Menschen abgelesen, ohne dass das Feldgerät selbst Teil eines automatisierten Kommunikationssystems ist.

Feldgeräte, die über Feldbusse kommunizieren, sind damit zwar grundsätzlich kommunikationsfähig, jedoch nicht im Sinne eines autonomen Netzwerkteilnehmers.
Die Kommunikation erfolgt typischerweise entweder über Punkt-zu-Punkt-Verbindungen (z.~B. klassische 4-20-mA-Schleifen) oder über Feldbusse, bei denen mehrere Feldgeräte gemeinsam an einem Bussegment betrieben werden.
Solche Segmente sind elektrisch und logisch klar abgegrenzt und werden über definierte Kopplungspunkte, etwa Ein-/Ausgangskarten oder Gateway-Module, an die darüberliegenden Steuerungs- oder Leitebenen angebunden.

Aus Sicht des einzelnen Feldgeräts bleibt die Kommunikationsschnittstelle dabei stets auf ein analoges Signal (4-20 mA) und/oder ein nicht-IP-basiertes Feldprotokoll beschränkt. Die Anbindung in IP-basierte Automatisierungs- oder IT-Netze erfolgt ausschließlich indirekt über die vorgelagerte Infrastruktur. Genau diese strukturelle Eigenschaft unterscheidet nicht-netzwerkfähige Feldgeräte grundlegend von modernen IoT- oder IIoT-Geräten und bildet die Ausgangsbasis für die Betrachtung sicherer Kommunikation und sicheren Onboardings in dieser Arbeit.
%%% Passt

% ## 2. Wie kommunizieren nicht netzwerkfähige Feldgeräte
% ### 2.1 Kommunikationskanälee
\subsection{Kommunikation bei nicht netzwerkfähigen Feldgeräten}

Nicht netzwerkfähige Feldgeräte kommunizieren in der Praxis über vergleichsweise einfache, feldnahe Übertragungsmechanismen, die historisch auf Robustheit, deterministisches Verhalten und lange Lebensdauern ausgelegt sind. Im Unterschied zu IP-basierten Endgeräten treten sie nicht als eigenständige Netzwerkteilnehmer auf, sondern sind aus Sicht der höheren Ebenen typischerweise über Kopplungskomponenten (z.~B. Ein-/Ausgangskarten, Remote-I/O oder Gateways) angebunden.

\todo{Was noch?}
%%% Passt

% ## 4. Onboardung und Geräteidentität
\subsection{Onboarding und Geräteidentität}
Im Folgenden wird beschrieben, wie nicht netzwerkfähige Feldgeräte heute in Anlagen aufgenommen (Onboarding) und im Betrieb eindeutig zugeordnet werden.

% ### 4.1
%% Onboarding Prozess
In der Praxis beginnt das Onboarding eines Feldgeräts meist mit der Zuordnung zwischen dem physischen Gerät und einer Messstelle bzw.\ einem Anlagentag.
Dazu werden typischerweise Typenschildinformationen (Hersteller, Typbezeichnung, Seriennummer) mit den Planungsunterlagen abgeglichen.
Diese Informationen werden dann im Asset Management System abgelegt.
Dort wird die physische Messstelle (z.\,B.\ Tag-Nummer) manuell mit Geräteinformationen verknüpft.
Dadurch können die Feldgeräte in Zukunt lokalisiert und identifiziert werden, sowie das Austauschen von Geräten, bzw\. das Aktualisieren von Geräten (z.~B. neue Firmware) koordiniert werden. \cite{stouffer_guide_2023}.
Diese Informationen die eine Identifikation unterstützen sollen, sind dabei aber nicht kryptografisch gesichert, sondern dienen lediglich der Identifikation und sicherstellung, dass das Gerät äußerlich dem entspricht, das man erwartet.

%% Security Maßnahmen
Da die Identität nicht netzwerkfähiger Feldgeräte hauptsächlich organisatorisch, und nicht kryptografisch, abgesichert ist,  muss auch organisatorisch sichergestellt werden,
dass es keinen unbefugten Zutritt zur Anlage bzw. zum Gerät gibt.
Wenn kryptografisch nicht sichergestellt werden kann, ob ein Gerät evt. unautorisiert getauscht bzw. manipuliert wurde, muss es organisatorisch sichergestellt werden.
Dafür werden physische Schutzmaßnahmen angewendet.
Zutrittskontrollen (Wer darf an die Anlage, Schaltschränke, Klemmenkästen), Zäune, verschlossene Technikräume oder Schränke. NIST SP~800-82 nennt physische und organisatorische Kontrollen als integralen Bestandteil eines OT-Sicherheitsprogramms, u.\,a.\ weil viele Angriffe und Fehlhandlungen in OT erst durch physischen Zugriff möglich werden \cite{stouffer_guide_2023}.
In der Praxis wird damit ein erheblicher Teil der Verantwortung für die Sicherstellung der Geräteintegrität und -identität auf Betreiberprozesse und physische Zugriffskontrolle verlagert.

Die beschriebenen Verfahren sind in der industriellen Praxis etabliert, haben jedoch systematische Grenzen. Insbesondere liefern Nameplate/Seriennummer, Dokumentationsabgleich und konfigurierbare Kennzeichen (z.\,B.\ Tag-Felder) keinen kryptografischen Beweis dafür, dass die Kommunikationsbeziehung tatsächlich mit dem erwarteten physischen Gerät endet. Konfigurierbare Identifikationsattribute können prinzipiell geändert oder nachgeahmt werden.

Zudem adressieren rein physische Schutzmaßnahmen Insider-Bedrohungen nur begrenzt: Personen mit berechtigtem Zugang können Geräte tauschen, manipulieren oder Parameter verändern, ohne dass dies zwangsläufig erkannt wird.
OT-Sicherheitsleitfäden behandeln solche Risiken unter anderem durch Forderungen nach kontrollierten Änderungen, Protokollierung und klaren Rollen/Prozessen, weisen aber zugleich darauf hin, dass organisatorische Maßnahmen allein keinen technischen Herkunftsnachweis des Geräts liefern \cite{stouffer_guide_2023}.
Für nicht netzwerkfähige Feldgeräte ergibt sich daraus eine Lücke zwischen praktischer Zuordnung (Asset-Verwaltung) und technischer, beweisbarer Geräteauthentizität.
Weiterhin gibt es auch Einsatzbereiche, die nicht physisch schützbar sind, da sie öffentlich zugänglich sind.
Beispielsweise Sensoren, die im Bereich Wastewater, also z.B. Kanalisation, eingesetzt werden. 

% 5 Warum gibt es keine Krypto?
\subsection{Warum gibt es keine Kryptografie?}
Dieses Kapitel ordnet ein, weshalb kryptografisch abgesicherte Kommunikation in nicht netzwerkfähigen Feldgeräten historisch nur eingeschränkt umgesetzt wurde und welche Entwicklungen diese Situation heute verändern.

Nicht netzwerkfähige Feldgeräte sind häufig für besonders robuste und energieeffiziente Betriebsbedingungen ausgelegt. Bei loop-versorgten 2-Draht-Geräten muss die gesamte Elektronik aus dem begrenzten Energiehaushalt der \SI{4}{\milli\ampere}--\SI{20}{\milli\ampere}-Stromschleife betrieben werden.
Abzüglich Toleranzen und Puffer, stehen 4...20mA Geräten ein Strom von ca. \SI{3.5}{\milli\ampere} zur Verfügung \cite{johnson_power_2013.}
Daraus resultiert, dass Mikrocontroller in Feldgeräten oft mit niedrigen Taktraten betrieben werden und die verfügbaren Ressourcen auf das für Messwerterfassung, Signalverarbeitung, Diagnose und Kommunikation notwendige Minimum optimiert sind.

Kryptografische Verfahren, insbesondere asymmetrische Verfahren sowie moderne, authentifizierte Verschlüsselung, sind in reiner Softwareausführung vergleichsweise rechenintensiv.
In ressourcenbeschränkten Feldgeräten führt dies typischerweise zu langen Ausführungszeiten und erhöhtem Energieverbrauch.
Für den Anlagenbetrieb kann dies problematisch sein, da zusätzliche Latenzen im Kommunikations- oder Parametrierpfad auftreten und gleichzeitig der ohnehin knappe Leistungsrahmen belastet wird.
In der Konsequenz wurden Sicherheitsmechanismen in vielen Feldgerätekategorien entweder gar nicht vorgesehen oder auf einfache Schutzfunktionen (z.\,B.\ Schreibschutz, PIN/Lock, organisatorische Prozesse) beschränkt \cite{bsi_-_bundesamt_fur_sicherheit_in_der_informationstechnik_ics_2024}.

Diese Situation wird durch die lange Nutzungsdauer industrieller Feldgeräte zusätzlich verstärkt. Feldgeräte verbleiben häufig über Zeiträume von 10 bis 15 Jahren (oder länger) im Betrieb.
Gerätewechsel sind kostenintensiv, erfordern Stillstände und sind durch Zertifizierungen, und qualitätssichernde Prozesse gebremst.
Dadurch existiert eine große installierte Basis an Legacy-Geräten, deren Hardwareplattformen nicht für moderne Kryptografie ausgelegt wurden. Selbst wenn neue Sicherheitsanforderungen entstehen, setzen sie sich in der Feldebene daher nur langsam durch \cite{bsi_-_bundesamt_fur_sicherheit_in_der_informationstechnik_ics_2024}.

%HW Krypto
In den letzten Jahren hat sich die Hardwarelandschaft jedoch deutlich weiterentwickelt. Moderne Mikrocontroller für Industrie- und Embedded-Anwendungen integrieren zunehmend dedizierte Krypto-Beschleuniger, etwa für AES, SHA und elliptische Kurvenverfahren (ECC). Ergänzend werden Sicherheitsfunktionen wie geschützte Schlüsselspeicher, sichere Bootketten, TrustZone-basierte Isolierung, manipulationsresistente Speicherbereiche oder externe Secure-Elemente verfügbar. Dadurch verlagern sich rechenintensive kryptografische Primitive in spezialisierte Hardwareblöcke, die sowohl schneller als auch energieeffizienter arbeiten als reine Softwareimplementierungen.
Beispielhafte Messungen für einen STM32U3-Mikrocontroller zeigen diesen Effekt deutlich:
Für AES-128 im Galois/Counter Mode (GCM) wird in der dedizierten Krypto-Hardware ein Datendurchsatz von etwa
9{,}17~\si{\mega\byte\per\second} erreicht, während eine reine Software-Implementierung auf demselben Controller lediglich etwa 0{,}76~\si{\mega\byte\per\second} erzielt.
Für SHA-256 liegen die gemessenen Durchsätze bei 45{,}87~\si{\mega\byte\per\second} in Hardware gegenüber 1{,}355~\si{\mega\byte\per\second} in Software \cite{oryx_embedded_benchmark_nodate}.
Somit ist die Verarbeitung in Hardware ca. 12- bzw.~ 34-mal schneller als in Software.
Während diese Werte natürlich von Controller, Krypo-Peripherie und Implementierung des Algorithmus abhängen, zeigen sie doch deutlich, um welche Größenordnung die Aktionen beschleunigt werden können.

Aus Systemsicht hat dies zwei Konsequenzen. Erstens wird Kryptografie unter den Randbedingungen der Feldebene überhaupt erst praktikabel, weil Energie- und Laufzeitkosten pro Operation sinken. Zweitens eröffnen sich dadurch neue Architekturoptionen: Auch ohne vollwertigen IP-Stack kann ein Gerät kryptografische Operationen, Schlüsselableitung und geschützte Datenübertragung realisieren, sofern ein zuverlässiger Byte-Transportkanal vorhanden ist. Damit werden auf IP basierende Konzepte prinzipiell auch über serielle oder proprietäre Feldschnittstellen denkbar, vorausgesetzt Protokollaufbau und Nachrichtenformate werden an die beschränkten Ressourcen angepasst.

Trotz der verbesserten Hardwarebasis bleibt eine wesentliche Lücke bestehen: Für viele nicht-IP-basierte Feldkommunikationswege existiert kein breit etablierter Standard, der eine kryptografisch eindeutige Geräteidentifikation bietet.
Das obwohl mittlerweile durch entsrpechende HW, die Möglichkeit kryptografische Operationen in akzeptabler Zeit durchzuführen, da ist.




%%%%%%%%%%%%%%%%%%%%%%%%%%%%%%%% Ende
\iffalse
% Auszug aus BSI, 3.2.1.4 Man in the middle Angriff - unverschlüsselte Kommunikationen
Die Hauptpriorität der Sicherheit von OT wird oftmals in der Verfügbarkeit und Zuverlässigkeit der Systeme gesehen.
Aspekte der Vertraulichkeit und Integrität werden unter Umständen nachrangig berücksichtigt.
Bspw. wird häufig auf eine Verschlüsselung der Daten oder Transportwege verzichtet.
Hieraus entsteht die Gefahr das Daten von Angreifern abgefangen sogar manipuliert werden können.
In einem solchen Fall ist die Integrität und die Vertraulichkeit der Daten nicht mehr sichergestellt. 
Ein Angreifer mit physischem Zugriff auf das OT-Netz kann diese Werte somit auslesen, verändern oder neue einspielen (z. B. zur Steuerung einer Maschine oder zur Fälschung von Sensordaten \cite{bsi_-_bundesamt_fur_sicherheit_in_der_informationstechnik_ics_2024}.

% Warum gibt es keine krypto?


% Wie wird geschützt?
% Air Gapped


% Thema lange Einsatzdauer
Ein weiterer Aspekt ist die lange Einsatzdauer industrieller Feldgeräte.
Komponenten der Betriebstechnik (OT) werden in industriellen Steuerungssystemen typischerweise über Zeiträume von 10 bis 15 Jahren oder länger betrieben, deutlich länger als klassische IT-Hardware (Zitat).
Das bedeutet, dass heute noch eine große installierte Basis von Feldgeräten mit älterer, nicht kryptofähiger Hardware im Feld ist.
Die Modernisierung hin zu Geräten mit integrierter Krypto-Hardware und damit die breite Umsetzung kryptografisch gesicherter Verbindungen bis hinunter zum Feldgerät erfolgt daher nur schrittweise im Rahmen von Migrations- und Retrofit-Projekten
und wird durch Lebensdauer, Zertifizierungen (z. B. ATEX/IECEx) und die hohen Kosten von Gerätewechseln zusätzlich verlangsamt.


% Thema begrenzter Strom
Bei vielen industriellen Feldgeräten, insbesondere bei Feldgeräten mit 2-Draht-Technik, ist der verfügbare Energiehaushalt stark begrenzt.
Der Strom für die gesamte Elektronik (Sensorik, A/D-Wandlung, Signalverarbeitung und Kommunikation) muss typischerweise aus wenigen Milliampere der Stromschleife bereitgestellt werden. 
Designrichtlinien wie \cite{johnson_power_2013} nennen für Feldgeräte Budgets von etwa 3~bis 3{,}5~\si{\milli\ampere} für die interne Elektronik, die nicht überschritten werden dürfen, damit der Messbereich von 4~bis 20~\si{\milli\ampere} eingehalten werden kann. %Pass

Kryptographische Verfahren, die rein in Software auf einem Mikrocontroller ohne spezielle Krypto-Peripherie ausgeführt werden,
sind im Vergleich zu klassischer Signalverarbeitung in der Regel deutlich rechen- und energieintensiver.
Die Ausführung von Beispielsweise AES oder ECC in Software, weist eine hohe Anzahl von Taktzyklen auf und entsprechend einen signifikanten Energiebedarf pro Operation, was sich unmittelbar auf Laufzeit und Leistungsaufnahme auswirkt. \todo{Brauch ich hier eine Quelle?}

In Feldgeräten, deren Taktfrequenz zusätzlich bewusst niedrig gewählt wird, um die Verlustleistung zu minimieren, müssen diese kryptographischen Operationen in das ohnehin sehr knappe Leistungsbudget eingepasst werden. 
Dies kann dazu führen, dass entweder die Rechenzeiten für Kryptofunktionen inakzeptabel lang werden, oder der zulässige Energieverbrauch überschritten würde. 
In der Praxis ist dies ein wesentlicher Grund dafür, dass viele existierende Feldgeräte bislang keine oder nur sehr eingeschränkt kryptographische Mechanismen unterstützen. \todo{Quelle finden}.
 
Ein etablierter Ansatz \todo{Formulierung}, um diesen Zielkonflikt zu entschärfen, ist der Einsatz dedizierter Krypto-Peripherie bzw.~ Hardwarebeschleuniger. 
Hierbei werden rechenintensive Primitive wie AES, SHA oder ECC in eigenständigen Hardwareblöcken implementiert, die speziell auf diese Operationen hin optimiert sind und deutlich weniger Taktzyklen sowie weniger Energie pro Operation benötigen als eine reine Software-Implementierung. %\cite{banerjee_energy-efficient_2017,panic_embedded_2016,rozlomii_hardware_2024}
Beispielhafte Messungen für einen STM32U3-Mikrocontroller zeigen diesen Effekt deutlich:
Für AES-128 im Galois/Counter Mode (GCM) wird in der dedizierten Krypto-Hardware ein Datendurchsatz von etwa
9{,}17~\si{\mega\byte\per\second} erreicht, während eine reine Software-Implementierung auf demselben Controller lediglich etwa 0{,}76~\si{\mega\byte\per\second} erzielt.
Für SHA-256 liegen die gemessenen Durchsätze bei 45{,}87~\si{\mega\byte\per\second} in Hardware gegenüber 1{,}355~\si{\mega\byte\per\second} in Software \cite{oryx_embedded_benchmark_nodate}.
Somit ist die Verarbeitung in Hardware ca. 12- bzw.~ 34-mal schneller als in Software.
Während diese Werte natürlich von Controller, Krypo-Peripherie und Implementierung des Algorithmus abhängen, zeigen sie doch deutlich, um welche Größenordnung die Aktionen beschleunigt werden können.
Somit werden auch auf eigentlich leistungsschwacher, energieoptimierter Hardware kryptographische Operationen in vertretbarer Zeit ausführbar, sobald geeignete Hardwarebeschleuniger vorhanden sind. \todo{Wie verhält sich der Energieverbrauch dabei?}


Ein zusätzlicher Vorteil integrierter Krypto-Peripherie liegt in der verbesserten Sicherheit des Gesamtsystems. 
Moderne Mikrocontroller für Industrie- und IoT-Anwendungen kombinieren Hardwarebeschleuniger für symmetrische und asymmetrische Kryptographie mit weiteren Sicherheitsfunktionen wie sicherer Schlüsselerzeugung, zertifizierten Entropiequellen, geschützten Schlüsselspeichern, Anti-Tampering-Mechanismen und sicheren Boot-Mechanismen.
Damit bilden sie die Grundlage dafür, auch in nicht IP-basierten Feldgeräten zertifikatsbasierte Identitäten und kryptographisch gesicherte Verbindungen zu realisieren, ohne die strengen Vorgaben an den Energieverbrauch und die Echtzeitfähigkeit zu verletzen.

\fi