%===================================================================
%%
%%  Hauptdatei  für Hinweise Abschlussarbeiten
%%
%%  Autor:   Arnemann
%%
%===================================================================
%% P R Ä A M B E L
%-------------------------------------------------------------------
%
% !TeX outputDirectory=build
% !TeX root = _iMain.tex
%==========================================================================================
%%
%% This is file 'FBMStyle.tex'  v 1.0
%% basieren auf einem Muster von Uwe Kappler 
%%
%%  2016-03-25 
%%========================================================================================== 
\documentclass[
        final,                % wenn es denn fertig ist.. 
        11pt,                 % Schriftgröße (10pt, 11pt, 12pt) 11:pala:ok, 
        twoside,              % oneside, twoside
        numbers=noenddot,     % Kapitelnummer ohne . am Ende
        headings=normal,      % Groesse der Ueberschrift (bigheadings, normalheadings, 
        parskip=half,         % Europäischer Satz mit Abstand zwischen Absätzen
        index=totoc,          % Index ins Verzeichnis einfügen	*** prüfen     
%        headsepline,          % Strich unter Kopfzeile
        headinclude,          % Kopfzeile bei Satzspiegel bereucksichtigen
        DIV=15,                % mit palatino 11pt oder CM 12
        BCOR=14mm]             % Binderand    %1mm, für Klebebindung % 14mm dann etwa gleichmäßig
   {scrbook}                  % Dokumenttyp // für große Berichte, Proejektarbeitebn, Abschlussarbeiten
%----------------------------------------------------------------------------------------
%  PACKAGES
%----------------------------------------------------------------------------------------
\typearea                  % nach der Schrift
        [current]          % Heftrand BCOR oben definiert
%        {calc}             % DIV neu berechnen aus package 
        {last}             % DIV letzte aus Definition zuvor

%XXXXXXXXXXXXXXXXxx
%TEST
%    XXXXXXXXXXX    
\setcounter{tocdepth}{2}
\setcounter{secnumdepth}{3}

\usepackage{lmodern}          % empfohlen anstatt CM Fonts, nur so korrekte Darstellung mit CM

\setkomafont{footnote}{\small}  % Marke und Text einer Fußnote, geht auch noch kleiner
                              % bis 2019 usepackage{scrpage2} 
\usepackage{scrlayer-scrpage} % Package laden  2019-09-22 

\usepackage[T1]{fontenc}      % T1-encoded fonts: auch Wörter mit Umlauten trennen
\usepackage[T1]{url}          % much like \verb allow line breaks for paths and URLs
\usepackage[utf8]{inputenc}   % Eingabe nach UTF-8  2016-09-16 

\usepackage{xspace}           % PS Bilder
\usepackage{graphicx}    
\usepackage{graphicx,color}   % JPEG und PNG 
                              % http://en.wikibooks.org/wiki/LaTeX/Colors 
                              
\usepackage{pdfpages}
                               
\usepackage{caption}	      % für subcaption, muss vorher geladen werden 2020-08-03 
\usepackage{subcaption}	      % Mehrere Bilder in einem mit ver. Bildunterschriften 2020-08-03 
\usepackage{wrapfig}	      % Text um Bild

\usepackage[ngerman]{babel}   % Neue deutsche Rechtschreibung  120412ar
%\usepackage[babel,german=guillemets]{csquotes} % für quotes
\usepackage[babel,german=quotes]{csquotes} % für quotes

%tools}             % enthält: afterpage, array, bm, calc, dcolumn, delarray, enumerate, fileerr, fontsmpl, ftnright, hhline, indentfirst, layout, longtable, multicol, rawfonts, showkeys, somedefs, tabularx, theorem, trace, varioref, verbatim, xr, and xspace.
\usepackage{float}

\usepackage{latexsym}         % Sonderzeichen  für ??
\usepackage{pifont}           % dito
\usepackage{array}            % für aufwändigere Tabellen
\usepackage{longtable}        % seitenübergreifende Tabellen passt zu KOMA
\usepackage{multicol}         % Mehrspaltiger Satz
\usepackage{paralist}         % zB für compactitem
\usepackage{tabularx}

\usepackage{makeidx}          % für Index-Erstellung 
%\usepackage{listings}         % für Latex Quelltext
%\usepackage{textcomp}         % for upright mu (\textmu)
\usepackage[fleqn]{amsmath}   % um Gleichungen linksbündig ggf. mit Einzug zu formatieren
\usepackage{amssymb}          % Symbole  
\usepackage{eqnarray}	        % nummerierte und unnummerierte Gleichungen/systeme
\usepackage{ifthen}           % logische Abfragen bei der Formatierung
\usepackage{fancybox}         % für schattierte ovale Boxen etc. geht nicht mit Miktex 5 060614ar
\usepackage{lscape}           % für landscape
\usepackage[normalem]{ulem}   % Unterstreichen und durchstreichen  Probleme mit bibtex
\usepackage{ziffer}           % Komma in math. Umgebung ohne folgendes Leerzeichen
\usepackage{cancel}           % Kürzen in Brüchen 
\usepackage{siunitx}          % für Zahlen und Einheitem 2016-03-09  
%\usepackage{wasysym}          % für \Square 

\usepackage{booktabs}         %  toprule midrule 

\usepackage{todonotes}

%% 2016-09-16 
\usepackage[backend=biber, language=ngerman, 
style=numeric, 
%citestyle=authortitle, 	%
%style=ieee,      % geht nicht
%citestyle=ieee, 	%
sorting=nyt,      % QuellenSortierung none  nyt: Name Titel Jahr     nty
giveninits=true,	% Vornamen abkürzen *** geht wenn Norm in Author steht und ~ 2017-07-31 
%firstinits=true,	% Vornamen abkürzen *** geht wenn Norm in Author steht und ~
maxbibnames=99, 	% Alle Autoren (kein et al.)
maxcitenames=1, 	% Kürzel nur aus 1. Autor
backref=true, 		% Rückverweise auf Zitatseiten
maxnames=50,
block=none]%
{biblatex}
%http://www.khirevich.com/latex/biblatex/

\ExecuteBibliographyOptions{%
	bibencoding=utf8, % wenn .bib in utf8
	bibwarn=true,     % Warnung bei fehlerhafter bib-Datei
}%
% style: numeric [1], alphbetic [Knu01], authoryear,
%
\DefineBibliographyStrings{ngerman}{andothers={et\ al\adddot}} % "u.a." zu "et al." 
\DefineBibliographyStrings{ngerman}{and={;}} % "und" zu ";"
\DefineBibliographyStrings{ngerman}{urlseen={online - zuletzt aufgerufen am}}
\DefineBibliographyStrings{ngerman}{bibliography={Literaturverzeichnis}}
\setcounter{biburllcpenalty}{7000}
\setcounter{biburlucpenalty}{8000}

%%\DeclareNameAlias{default}{last-first}  % erst Nachname, Vorname: dann wird Normeintrag zerruppt
\DeclareNameAlias{default}{family-given}  % 2019-09-23 

\usepackage[plainpages=false,pdfpagelabels]{hyperref}         
%
%% **** END OF CLASS MStyle ****

%schwierigkeit  numerisch nach normzahlen klappt nicht

%\usepackage[
%style=authoryear-icomp,    % Zitierstil
%isbn=false,                % ISBN nicht anzeigen, gleiches geht mit nahezu allen anderen Feldern
%pagetracker=true,          % ebd. bei wiederholten Angaben (false=ausgeschaltet, page=Seite, spread=Doppelseite, true=automatisch)
%maxbibnames=50,            % maximale Namen, die im Literaturverzeichnis angezeigt werden (ich wollte alle)
%maxcitenames=3,            % maximale Namen, die im Text angezeigt werden, ab 4 wird u.a. nach den ersten Autor angezeigt
%autocite=inline,           % regelt Aussehen für \autocite (inline=\parancite)
%block=space,               % kleiner horizontaler Platz zwischen den Feldern
%backref=true,              % Seiten anzeigen, auf denen die Referenz vorkommt
%backrefstyle=three+,       % fasst Seiten zusammen, z.B. S. 2f, 6ff, 7-10
%date=short,                % Datumsformat
%]{biblatex}
%\setlength{\bibitemsep}{1em}     % Abstand zwischen den Literaturangaben
%\setlength{\bibhang}{2em}        % Einzug nach jeweils erster Zeile



%\ExecuteBibliographyOptions{firstinits=true}% Vornamen abkürzne
%\ExecuteBibliographyOptions{%
%maxbibnames=99, % Alle Autoren (kein et al.)
%maxcitenames=1, % Kürzel nur aus 1. Autor
%backref=true, % Rückverweise auf Zitatseiten
%}%

%
%%----------------------------------------------------------------------
%% pdf-setup 060830ar
%%----------------------------------------------------------------------
%
%
\hypersetup{%
  %backref,%
  pdftitle    = {Titel der Thesis},
  pdfsubject  = {Fakultät für Elektro- und Informationstechnik},
  pdfauthor   = {Kilian Rupp},
 % pdfkeywords = {Richtlinie, Abschlussarbeit, Bericht},
%  pdfcreator  = {Adobe-Acrobat-Distiller},
%  pdfproducer = {LaTeX with hyperref und thumbpdf},
  pdfpagemode  = UseThumbs, % Anzeige Piktogramme
  bookmarksopen = true,     % Anzeige aller Ebenen
  bookmarksnumbered = true, % Anzeige Abschnittsnummern  
  pdfstartpage = {1}        % Startseite, hilfreich mit pdf
 }
% 

\hypersetup{%
%  pagebackref,
  bookmarksnumbered,
%  backref,
%  bookmarks,
%  breaklinks,
%  linktocpage,
%
%alternative Farbwahl
%  colorlinks  = true,
%  linkcolor   = blue,
%  urlcolor    = blue,
%  tocdepth     = 3            
}
             
%%
%% wenn für alles die gleiche Farbe ..
%%
%\definecolor{LinkColor}{rgb}{0,0,0.5} %duckelblau
\definecolor{LinkColor}{rgb}{0,0,0} % schwarz
%\definecolor{LinkColor}{rgb}{0,0,1} %blau
%\definecolor{LinkColor}{rgb}{0,1,0} %green
%\definecolor{LinkColor}{rgb}{0.5,0,0} % dunkelrot bis braun
%\definecolor{LinkColor}{rgb}{1,0,0} % rot
%\definecolor{LinkColor}{rgb}{0.7,0,0} % kräftig rot
\hypersetup{%
  colorlinks=true,%
	linkcolor=LinkColor,%
	citecolor=LinkColor,%
	filecolor=LinkColor,%
	menucolor=LinkColor,%
%	pagecolor=LinkColor,%  gibt es nicht mehr
	urlcolor=LinkColor
}
      % z.B. Farben für Links  

%#############################################################################
%
% Makros       Arnemann
%
% 2015-10-18ar
% 2016-03-02 
%%=============================================================================
%%-- Farben 
%%=============================================================================
%%
%%% Farbe .s http://www.namsu.de/Extra/pakete/Xcolor.html
%%%  Hochschulfarbe
%\definecolor{hskarot}{rgb}{219,0,49} 
%\definecolor{hskarot}{rgb}{0.86,0,0.192} % 
\definecolor{hskarot}{rgb}{0.7,0,0} % kräftig rot Ersatz für Hska orig

%\colorlet{showcolor}{white}       % zum Verbergen von Texten  z.B. in ToDo-boxen
                                   % Text wird aber gedruckt, man kann danach suchen 
%\colorlet{showcolor}{red}          % zum Verbergen von Texten  z.B. in ToDo-boxen
%
%%-----------------------------------------------------------------------------
%% Neue Umgebungen
%%-----------------------------------------------------------------------------
%
% --- Gleichungen 
% Syntax: \beq{NAME DER GLEICHUNG} 
%         \eeq
% Referenz: \ref{eqt:NAME DER GLEICHUNG}
%
\newcommand{\beq}[1]
           {
            \begin{equation}
            \label{#1}
           }
% end
\newcommand{\eeq}
           {
             \end{equation}
           }
\newcommand{\ba}{\begin{array}}
\newcommand{\ea}{\end{array}}

\newcommand{\bdm}{\begin{displaymath}}
\newcommand{\edm}{\end{displaymath}}
	   
% --- Itemize 
\newcommand{\bi}{\begin{itemize}}
\newcommand{\ei}{\end{itemize}}

% --- Itemize 
\newcommand{\bci}{\begin{compactitem}}
\newcommand{\eci}{\end{compactitem}}


% --- Enumerate
\newcommand{\be}{\begin{enumerate}}
\newcommand{\ee}{\end{enumerate}}

% ---
\newcommand{\bd}{\begin{description}}
\newcommand{\ed}{\end{description}}

% --- Umgebungen wie compactitem, compactenum etc
% erlauben kleinen Abstand zwischen Aufzählungen


% --- Umgebung
%\def\d@nger{\marginpar[\hfill\dbend]{\dbend\hfill}}
%\newenvironment{danger}{\medskip\hspace{0pt}\d@nger}{\medskip}
%--- Umgebung  2015-05-09  diese Definition mit aen ist besser als mit @ . Sonst erhebl Probleme
\def\daenger{\marginpar[\hfill\dbend]{\dbend\hfill}}
\newenvironment{danger}{\medskip\hspace{0pt}\daenger}{\medskip}

%%-----------------------------------------------------------------------------
%% Abkürzungen 
%%-----------------------------------------------------------------------------
%
%--- Fußnoten
% Syntax: \fn legt fest, wo das Fußnotenzeichen steht
%         \fnt{FUSSNOTENTEXT} legt den Text fest
\newcommand{\fn}{\footnotemark}
\newcommand{\fnt}[1]{\footnotetext{#1}}

\newcommand{\ol}{\ddot{O}l}
\newcommand{\p}{\partial}
\newcommand{\Dp}{\Delta p}
\newcommand{\te}{$\vartheta$}             % spezielles theta für Temperatur Celsius
\newcommand{\cT}{\vartheta}               %      Celsiustemperatur, dass nicht in  t geändert werden darf
\newcommand{\Ct}{\vartheta}               %var   Celsiustemperatur, kann evtl in t geändert werden
\newcommand{\R}{{\em\bf R}}               % Universelle Gaskonstante fett 
%%
%% --- Einheiten werden vorzugsweise mit dem package siunitx formatiert !
%%
\newcommand{\C}{$^\circ$C}                % Grad Celsius als Einheit im Text
%\newcommand{\C}{~\textcentigrade{}}      % Grad Celsius alternativ
%\newcommand{\CC}{^\circ\mbox{C}}         % Grad Celsius alternativ2
\newcommand{\CC}{\,^\circ\mathrm{C}}      % Grad Celsius als Einheit im mathemodus
\newcommand{\CCe}{^\circ\mathrm{C}}       % Grad Celsius als Einheit im mathemodus
%%
%% MatheOperatoren
%%
\newcommand{\mue}{\textmu}                % 
\renewcommand{\d}{\partial\mspace{2mu}}   % partielles Diff. Zeichen 
\newcommand{\td}{\,\mathrm{d}}           	% totales Diff (d, nicht kursiv)
\newcommand{\ddt}[1]{\frac{\td #1}{\td t}}% zweifach 
%% siehe
%% from physic  shttp://www.dfcd.net/articles/latex/latex.html
%%

% ---  unbestimmtes Differenzial  (kleines d mit horizontalem Strich durch
\def\dbar{{\mathchar'26\mkern-11mu\mathrm{d}}}  % hier für lmodern,  Achtung passt nich bei jedem Font !
%\def\dbar{{\mathchar'26\mkern-12mu d}}  %(The space after the \mu" is optional but is added f
%\def\dbar{{\mathchar'26\mkern-9mu\mathrm{d}}}  % 130102ar hier für times !
%%%
%% Index: oben oder/und unten
%%
\newcommand{\idx}[1]{_\mathrm{#1}}        % nicht kursiver Index in Gleichungen, mit Umlauten wie "a
\newcommand{\idxi}[2]{_{\mathrm{#1,}{#2}}}% 1. Parameter nicht kursiv, zweiter kursivzB für Laufvariable
\newcommand{\idy}[1]{^\mathrm{#1}}        % hochgestellt

%%
%% Texthervorhebungen
%%
\newcommand{\name}[1]{\textsc{#1}}        % für Firmen, Autoren
\newcommand{\uu}[1]{\emph{#1}}            % Texthervorhebung (Unterstreichungen sind unüblich)%%
\newcommand{\fettA}[1]{{\sffamily\small\textbf{#1}}}   % Texthervorhebung fett

\newcommand{\s}{\scriptscriptstyle}
\newcommand{\D}{\displaystyle}            % für Fonts in Brüchen

\newcommand{\bff}[1]{\noindent {\textbf{#1}}}

\newcommand{\du}[1]{\underline{\underline{#1}}} %  Ergebnisse von Berechungen doppelt unterstreichen

%%
%% Abkürzungen im Text, alphabetisch, 
%%
\def\bzw{bzw.\ }
\def\bspw{bspw.\ }
\def\ca{ca. }
%\def\CO2{CO$_2$}  % neu
\def\dh{d.\,h.\ }
\def\etc{etc.\ }
\def\evtl{evtl. }
\def\ggf{ggf.\ }
\def\inkl{inkl.\ }
\def\oä{o.\,ä.\ }
\def\og{o.\,g.\ }
\def\so{s.\,o.}
\def\su{s.\,u.}
\def\ua{u.\,a.\ }
\def\zB{z.\,B.\ }
\def\zT{z.\,T.\ }

%%
%%---------------------------------------------------------------
%%   Abkürzungen für Variablen in Gleichungen
%%---------------------------------------------------------------
%%
%% verkürzte Schreibweise, mathematische Symbole, alphabetisch, Beispiele
%%   wichtige Regel: keine Zahlen in Variablennamen 
%% 
\def\mp{\dot{m}}
\def\md{\dot{m}}                      %
%\def\m.{\dot{m}} % geht?
\def\pWsT{p\idx{W}\idy{S}(T)}         % 
\def\Vd{\dot{V}}                      %  

\def\hepx{h_\mathrm{1+X}}             % 

%%
%% Kennzahlen werden auch kursiv formatiert, aber mit wenig Zwischenraum, darum so:
%%  muss in $ $ 

\def\GWP{\mathin{GWP}}  % 
\def\Re{\mathin{Re}}    % Reynold
\def\Nu{\mathin{Nu}}    % Nusselt
\def\Pr{\mathin{Pr}}    % Prandtl
%%
%% --- Referenzen im Text -----------------------------------------------
%%
%  Referenzieren eine Tabelle im Text, beim ersten Mal
\newcommand{\RefTab}[1]{{\small \color{LinkColor}\raisebox{1pt}{$\blacktriangleright$}}{Tabelle~\ref{#1}}} 
%\newcommand{\RefTab}[1]{$\underline{\mbox{Tab.~\ref{#1}}}$}   
%\newcommand{\RefTab}[1]{\textbf{Tabelle~\ref{#1}}}            
%
%
%  Referenzieren eine Abbildung im Text: sollte mit RefTab identisch sein
\newcommand{\RefFig}[1]{{\small \color{LinkColor}\raisebox{1pt}{$\blacktriangleright$}}{Bild~\ref{#1}}} 
%\newcommand{\RefFig}[1]{{\color{blue}$\blacktriangleright$\textbf{Bild~\ref{#1}}}} 
%\newcommand{\RefFig}[1]{\textbf{Bild~\ref{#1}}}                 
%
%  Referenzieren eine Tabelle im Text, nach dem ersten Mal: sollte nicht hervorgehoben sein
\newcommand{\RefTabc}[1]{Tabelle~\ref{#1}}                     % 2. Tabellenref. im Text
\newcommand{\RefFigc}[1]{Bild~\ref{#1}}                        % 2. Bildref.
%
%  Referenzieren einer Gleichung 
\newcommand{\RefEq}[1]{\mbox{Gl.~(\ref{#1})}}                  % Gleichungen

%%
%% --- Einheiten werden immer steil formatiert! 
%%     Einheiten werden vorzugsweise mit dem *package siunits* formatiert !
%%     folgende Abkürzungen benötigen das Package nicht 
%%
\newcommand{\ZmE}[2]{$#1$~{#2}}                                % Zahl:#1 mit Einheiten#2 #3 in TextUmgebung
\newcommand{\zme}[2]{#1~\mbox{#2}}                             % Zahl:#1 mit Einheiten#2 #3 in MatheUmgebung
\newcommand{\EH}[1]{#1}                                        % Einheiten
\newcommand{\B}[1]{\mbox{#1}}                                  % Steiler Text in MatheUmgebung 
\newcommand{\eb}[2]{\frac{\text{#1}}{\text{#2}}}               % EINHEITEN in echten Brüchen

%%
%%--- Font Größe  ----------------------------------------------------------
%%
\newfont{\ssf}{cmss10 scaled 1000}
\newfont{\ssb}{cmssbx10 scaled 1000}
\newcommand{\cf}{\ssf}                 % CaptionFonts: in Bildunter-, Tab-Überschriften
\newcommand{\rf}{\em}                  % RefFonts    : Kennzeichnung von Referenzen im Text
\newcommand{\eng}{\ttfamily}           % englische Begriffe im deutschen Text besonders hervorheben
%
% Hinweis
% Syntax: \oops{ÜBERSCHRIFT}{TEXT}
%        : nach Überschtift wird automatisch eingefügt
\newcommand{\oops}[2]{\begin{quote}\textbf{#1}:\\ {#2} \end{quote} }

%---------------------------------------------------------------
% Bildunter- und Tabellenüberschriften
%---------------------------------------------------------------
% z.B. andere Schrift, oder auch Schriftform und andere Abkürzung
%
% alternaiv : "Abbildungen" lassen und
%  Zeilenumbruch bei Bildbeschreibungen einführen \setcapindent{1em}
% bessere Methoden: siehe Komaskript
%
\addto\captionsngerman{% 
\renewcommand{\figurename}{Bild}   % Standard wäre Abbildung
\renewcommand{\tablename}{Tabelle}    % Standard wäre Tabelle
} 
%
\renewcommand*{\captionformat}{.~} % DIN lässt grüssen
%\addtokomafont{caption}{\sffamily\small\raggedright}  % kleinere Schrift, linksbündig
\addtokomafont{caption}{\sffamily\small}  % kleinere Schrift in Überschrift
%\setkomafont{descriptionlabel}{\sffamily\small}
\setkomafont{captionlabel}{\sffamily\bfseries}
\setcaphanging
\setcapindent{0em}             % kein Einzug
%
%---------------------------------------------------------------
%  Kommentare
%---------------------------------------------------------------
%-- Formatieren eigener Kommentare, Anmerkungen, Arbeitsanweisungen 
\newcommand{\Kommentar}[1]{{\em #1}} 

%---------------------------------------------------------------
%-- Verstecktes
% Alles innerhalb von \Hide{} oder \ignore{}
% wird von LaTeX komplett ignoriert (wie ein Kommentar)
%------------------------
% methode 1
\newcommand{\Hide}[1]{}
\let\ignore\Hide
%
% methode 2
%\newcommand{\Hide}[1]{
%{\color{cyan}\sffamily\small #1}} % green red blue magenta cyan
%
% methode 3
\newcommand{\HideB}[1]{}
\let\ignore\HideB

%-----------------------------------------------------
% auskommentieren von längeren Textpassgen 
%
%\newcommand{\showme}{1}  % \usepackage{ifthen}
\newcommand{\hideandshow}[1]{%
	\ifthenelse{\isundefined{\showme}}{}{#1}}


%--- Umbrüche
%  eigner Befehl, lässt sich besser modifizieren
%  eigene Umbrüche, die nur bei der Textentwicklung eingeführt wurden,
%  können hier zentral entfernt werden
\newcommand{\mynewpage}{\newpage}
\newcommand{\myclearpage}{\clearpage}

%%====================================================================
%%
%%--  Tabellen 
%%
%\usepackage{tabularx} % automatische Spaltenbreite 
\newcolumntype{L}[1]{>{\raggedright\arraybackslash}p{#1}}
\newcolumntype{C}[1]{>{\centering\arraybackslash}p{#1}}
\newcolumntype{R}[1]{>{\raggedleft\arraybackslash}p{#1}}

\newcommand{\LCIRC}[1]{{\large\textcircled{\normalsize \texttt{#1}}}}
\newcommand{\NCIRC}[1]{\textcircled{ {\small \texttt{#1}} } }

\newcommand{\bul}{$\bullet$}              % Mark in Tabs
\newcommand{\Q}{$\bullet$}                % mark2 in Tabs

\newcommand{\mc}{\multicolumn}
\newcommand{\cg}{\centering}    % for multicolumn


%%%%%%%%%%%%%%%%%%%%%%%%%%%%%%%%%%%%%%%%%%%%%%%%%%%%%%%%%%%%%%%%%%%%%%%%%%%%%%%%%%%%%
%%% EOF  %%%%%%%%%%%%%%%%%%%%%%%%%%%%%%%%%%%%%%%%%%%%%%%%%%%%%%%%%%%%%%%%%%%%%%%%%%%%
%%%%%%%%%%%%%%%%%%%%%%%%%%%%%%%%%%%%%%%%%%%%%%%%%%%%%%%%%%%%%%%%%%%%%%%%%%%%%%%%%%%%%
         % Macros lagen
%-----------------------------------------------------------------
% Beschriftung des Titelblatts - Vorderseite
% 2016-09-15 Ar
%
% Alternativ: TitelBlatt und Rückseite mit anderem Programm erstellen
% dabei haben Sie viel mehr  Gestaltungsfreiheiten 
% dann pdf erzeugen und schließlich mit der Text-pdf zusammenführen
%-------------------------------------------------------------------
%
\usepackage{titlepage-HsKa/HsKAtitle11}
%\usepackage{titlepage-HsKa/HsKAtitle11-mit-logo} passt noch nicht

\department{Fakultät für Elektro- und Informationstechnik}

%Titel, Thesistitel
\title{Sicherer Verbindungsaufbau für nicht netzwerkfähige Feldgeräte auf Basis von Zertifikaten}

\thesistype{Masterthesis (M. Sc.)}  
%           *Bachelorarbeit B.Eng.  
%           *Masterthesis (M.Sc.)      
%\thesisno{Will ich entfernen}          % siehe Anmeldeformular     

\author{Kilian Nikolaus Rupp}
\BDate{06.01.1998}
\BLocation{Saarlouis}
\matno{67723}


\companyname{Hager Group}
\tutor{M. Sc. Nils Schlegelmilch}{Prof. Dr.-Ing. Philipp Nenninger\\ Prof. Dr.-Ing. Reiner Kriesten}
\location{Karlsruhe}
\duration{01.10.2025  bis  31.03.2026}

%\restrictionnotice{*S P E R R V E R M E R K}      
%\restrictionnotice{*** V E R T R A U L I C H ***}      
%%
% Bemerkung auf der Rückseite, wenn zweiseitig
%
\Schriftsatz{%
    Folgende Angaben sind optional (!)\\
    Satz und Herstellung:\\
    {\LaTeX} und KomaSkript  /  MiKTeX 2.9, TeXnicCenter 2.02 \\
    Font: Computer Modern, 11 pt\\
    Druckdatum: \today
    }%
    
%      % Titeldefinition, kann so nicht 
                                   % genutzt werden

%--- Bib.Datei einbinden
%\addbibresource{Literatur.bib}     % 
\addbibresource{bib.bib}
%	\hyphenation{}
%  	Worte mit Umlauten müssen im Text mit \- getrennt werden!
%
\hyphenation{En-thal-pie}
\hyphenation{En-thal-pie-dif-fe-renz}
\hyphenation{Koeffi-zienten}
%
%
% 	Wörter mit Bindestrichen trennen: Im Text mit "= den Bindestrich setzen!
%
%
%
% Hurenkinder und Schusterjungen verhindern
% Latex vergibt Punkte für Dinge die typographisch nicht korrekt sind. Je stärker ein "Fehler" bewerter wird, desto wahrscheinlicher ist es, dass Latex versucht diesen zu beheben. Mit Nachfolgendem Kommando, wird die Punktzahl für Hurenkinder und Schuterjungen erhöht:
% Achtung: Funktioniert nur mäßig, also besser nicht verwenden!!
%\clubpenalty10000
%\widowpenalty10000
%\displaywidowpenalty=10000
%	Alternative dazu ist \pagebreak, funktioniert erfahrungsgemäß ganz gut!   %	Defintionen für Silbentrennung

\makeindex

%===================================================================
% B E G I N N  D E S  D O K U M E N T S
%
%% Hier kommen die einzelnen Kapitel, die jeweils in einem eigenen
%% Folder stehen. Dateiendung MUSS .tex sein. Wird hier nicht angegeben. 
%% W I C H T I G : Pfade in Unix-Syntax d.h.:
%% 0010/chapter und NICHT 0010\chapter !!!
%% ALLE PFADE RELATIV ZU DIESER DATEI!!
%-------------------------------------------------------------------
\begin{document}

%---------------------------------------
% für Einheiten nach  siunitx
\sisetup{
output-decimal-marker = {,},
exponent-product = \cdot,  % für .10^7
}%
%----------------------------------------

\pagestyle{scrheadings}            % Kopf- und Fusszeilen
\clearscrheadfoot                  % Alles auf "" setzen
%\pagestyle{empty}

\hypersetup{pageanchor=false}
\begin{titlepage}
\maketitle
\end{titlepage}

%%
%%  Pagestyle für Pre
%%  061218ar
%==============================================================================
%
\pagestyle{scrheadings}            % Kopf- und Fusszeilen
\clearscrheadfoot                  % Alles auf "" setzen
\setcounter{page}{100}             % anpassen: wenn 1-seitig  dann 2     
\pagenumbering{roman}
%
%%bis 2017-09-24 
\lehead{\pagemark \hspace*{2em} \headmark}
\rohead{\headmark \hspace*{2em} \pagemark}

% ab 2017-09-24 
%%  Kolumnentitel  %%%%%%%%%%%%%%%%%%%%%%%%%%%%%%%%%%%%%%%%%%%%%%%%%%%
%%  alternativ: von wiley 
%%
\clearscrheadings
\clearscrplain
\pagestyle{scrheadings}

% Kopfzeile links gerade 
\lehead[%
  \llap{%
    \pagemark
    \hspace{1mm}%
    \smash{\rule[-2.8mm]{1pt}{6mm}}%
    \hspace{2mm}%
  }%
]{%
  \llap{%
    \pagemark
    \hspace{1mm}%
    \smash{\rule[-2.8mm]{1pt}{6mm}}%
    \hspace{2mm}%
  }%
  {%
    \sffamily
    \itshape
    \selectfont
    \headmark
  }%
}
% Kopfzeile rechts ungerade
\rohead[%
  \rlap{%
    \hspace{2mm}%
    \smash{\rule[-2.8mm]{1pt}{6mm}}%
    \hspace{1mm}%
    \pagemark
  }%
]{%
  {%
    \sffamily
    \itshape
    \selectfont
    \headmark
  }%
  \rlap{%
    \hspace{2mm}%
    \smash{\rule[-2.8mm]{1pt}{6mm}}%
    \hspace{1mm}%
    \pagemark
  }%
}

\renewcommand\pnumfont{%
  \sffamily%
%  \bfseries  %fette Schrift
  \upshape
%  \fontsize{8}{12}%
  \fontsize{10}{12}%
  \selectfont
}

\renewcommand\headfont{%
  \sffamily%
  \itshape
%  \fontsize{8.5}{12}%
  \fontsize{10}{12}%
  \selectfont
}
%============================================================================
     % Titelblatt erzeugen
%%
%% Erklärung und Sperrvermerk
%%
%% Vorsicht beim Entfernen von \\ und Leerzeilen
%%
%% Sperrvermerk nur wenn zwingend !
%%------------------------------------------------------------------------
%
\setcounter{page}{3}       % anpassen: wenn 1 Seitig  dannn 2     
\pagestyle{scrheadings}    % Kopf- und Fusszeilen

\chapter*{}
\centerline{\Large \textsf{\textbf{Erklärung}}} \label{Erklaerung}%\\
\vspace*{2ex}
%
Ich versichere hiermit wahrheitsgemäß, die Abschlussarbeit selbstständig angefertigt, alle benutzten Hilfsmittel vollständig und genau angegeben und alles einzeln kenntlich gemacht zu haben, was aus Arbeiten anderer unverändert oder mit Abänderungen entnommen wurde.\\
\\
\\

\vspace*{4ex}
Karlsruhe, den \today \\

\vspace*{4ex} 
Unterschrift:\\


\vspace{2cm}

%--------------------------------------------------------------
%\section*{}
%\centerline{\Large \textsf{\textbf{S p e r r v e r m e r k}}} %\\

%Diese Arbeit enthält vertrauliche (oder:  geheime) Informationen. 
%Veröffentlichungen insbesondere vor dem (*31.03.2013) 
%bedürfen der schriftlichen Genehmigung der (Organisation).

%\raggedbottom
%\cleardoublepage
%**************************************************************




         % Hochschul-spezifisch
%\include{001_Vorspann/danksagung}
\cleardoublepage
%%
%%
%%
%\thispagestyle{empty}
%\clearscrheadfoot                  % Alles auf "" setzen
%------------------------------------------------------------------------------

\section*{Sicherer Verbindungsaufbau für nicht netzwerkfähige Feldgeräte auf Basis von Zertifikaten} \label{Kurzfassung}
%Worum geht es
Die steigende Komplexität der Gebäudeautomatisierung und die Integration von Smart-Home-Systemen
haben die Anforderungen an Schalteinrichtungen für die Hausinstallationstechnik deutlich erhöht.
Diese Arbeit bietet eine umfassende Analyse und Vergleich der von der Firma Berker GmbH \& Co. KG speziell für die Unterputzmontage entwickelten elektronischen Schaltertypen.
Hierbei werden Schaltungen mit bistabilen Relais, MOSFETs und Triacs betrachtet.

%Wie wurde vorgegangen
Nach einer einführenden Darstellung der Problemstellung definiert die Arbeit die Anforderungen an \SI{230}{\volt} Schalteinrichtungen,
die sowohl schaltungstechnische als auch wirtschaftliche Aspekte umfassen. Diese Anforderungen werden durch Berechnungen, Simulationen und Messungen vergleichend untersucht. Darüber hinaus wird ein detaillierter Einblick in die Funktionsweise der Schaltungen gegeben, um den Lesern ein umfassendes Verständnis des Themas zu vermitteln.

%Ergebnisse
Die Ergebnisse zeigen, dass das bistabile Relais die besten schalttechnischen Eigenschaften bietet,
jedoch gleichzeitig wirtschaftlich weniger effizient ist. Die Triac-Schaltung ist zwar kostengünstig und einfach,
weist jedoch die schlechtesten elektrischen Eigenschaften auf.
Im Gegensatz dazu bietet die MOSFET-Schaltung einen ausgewogenen Kompromiss zwischen wirtschaftlichen und technischen Aspekten.

%Abschluss
Insgesamt beleuchtet diese Arbeit die komplexen Dynamiken der elektronischen Schalttypen für die Unterputzmontage
und liefert wertvolle Einblicke für Endverbraucher, Installateure und Entwickler im Bereich der Hausinstallationstechnik.
Durch ihre sorgfältige Analyse und den Vergleich trägt die Arbeit dazu bei, das Verständnis für die Leistungsfähigkeit
und Eignung verschiedener Schalttypen in unterschiedlichen Anwendungskontexten zu vertiefen.


\newpage

\section*{Secure Conncetion Establishment for Non-Network-Enabled Field Devices Based on Certitificates} \label{Abstract}
The increasing complexity of building automation and the integration of smart home systems have significantly raised the demands for switchgear in home installation technology. This paper provides a comprehensive analysis and comparison of electronic switch types specifically developed for wall-mounting by Berker GmbH \& Co. KG. These include circuits with bistable relays, MOSFETs, and Triacs.

After an introductory overview of the problem statement, the paper defines the requirements for \SI{230}{\volt} switching devices, encompassing both circuitry and economic aspects. These requirements are comparatively examined through calculations, simulations, and measurements. Furthermore, a detailed insight into the functioning of the circuits is provided to offer readers a comprehensive understanding of the subject.

The results indicate that bistable relays offer the best switching characteristics but are less economically efficient. While the Triac circuit is cost-effective and straightforward, it exhibits the poorest electrical properties. In contrast, the MOSFET circuit presents a balanced compromise between economic and technical aspects.

Overall, this study sheds light on the complex dynamics of electronic switching types for wall-mounted installations and provides valuable insights for end users, installers, and developers in the field of home installation technology. Through its careful analysis and comparison, the study contributes to deepening the understanding of the performance and suitability of various switching types in different application contexts.

\cleardoublepage        % 
%%==============================================================
%
% Nomenklatur
%
%==============================================================
\chapter*{Nomenklatur%
  \footnote{ Das ist hier nur beispielhaft; es werden sonst nur (alle) tatsächlich genutzte Zeichen aufgeführt!\\
	beim Sortieren: alphabetisch, erst kleine, dann große Buchstaben.} }%
  \label{Nomenklatur}%
   % Die folgende Zeile erzwingt einen Eintag ins Inhaltsverz.
  \addcontentsline{toc}{chapter}{Nomenklatur}%
%-----------
\manualmark
\markright{Nomenklatur}
\markleft{Nomenklatur}
%\ohead[]{\headmark}
%\begin{longtable}{p{3cm} p{2cm} p{\textwidth}}  % ACHTUNG Breite
\begin{longtable}{p{0.15\textwidth} p{0.15\textwidth} p{0.6\textwidth}}  % ACHTUNG Breite 0.9
%-----------------------------------------------
\multicolumn{3}{l}{%
\textbf{\textsf{\large Lateinische Formelzeichen}}
}\\
%-----------------------------------------------
%
%----Große Buchstaben----
&&\\
%----kleine Buchstaben----
$c\idx{lim}$              & {-}             & Konstante des Turbulenzmodells   \\
$g_i$                     & {m/s$^2$}       & Vektor der Schwerkraft    \\
%$k$                       & {m$^2$/s$^2$}   & turbulente kinetische Energie \\
$k$                       & \si{m^2/s^2}    & turbulente kinetische Energie \\
$l$                       &   m             & Längenmaß \\
$L$                       &  m              & charakteristische Länge \\
$\dot m$                  &   kg/s          & Massenstrom \\
$p$                       &   N/m$^2$       & Druck \\
$R$                       &   -             &  Residuenmatrix   \\
{\textbf{R}}              & \si{J/(mol.K)}  & Universelle Gaskonstante \\
$s$ 										  &  \si{J/(kg.K)}    & spezifische Entropie \\
$ S $                     & J/K             & Entropie \\
$S_{ij}$                  &   1/s           &  Scherrate   \\
$t$                       &   s             & Zeit \\
$ T $                     & K               & Temperatur \\
$u,v,w$                   &   m/s           & Geschwindigkeitskomponente \\
                      & & \\ % LEERZEILE

%----Griechische Buchstaben-----------------------------------------------------------------------
\\
\multicolumn{3}{l}{%
\textbf{\textsf{\large Griechische Formelzeichen}}
 }\\
$\alpha$                &    -                & allgemeine Zustandsgröße \\
$\beta^*$               &    -                & Konstante des Turbulenzmodells \\
$\eta$                  & \si{kg/(m.s)}       & dynamische Viskosität\\  
$\eta$                  &    W/W              & Wirkungsgrad\\  
$\vartheta$             &    \C               & Temperatur\\ 
$\nu$                   &  \si{m^2/s}         & kinematische Viskosität \\
$\varkappa, \kappa$     &     -               & Isentropenexponent \\ 
$\xi$                   &    kg/kg            & Massenanteil\\
$\varrho$               &  \si{kg/m^3}        & Dichte   \\
$\sigma_i$              &    -                & Konstante des Turbulenzmodells \\
$\tau_{ij}$             &  \si{N/m^2}         & Spannungstensor   \\
$\varphi$               &    -                & relative Feuchte \\
$\Phi$                  &    -                & allgemeine Konstante des Turbulenzmodells   \\
$\omega$                &    1/s              & Frequenz der turbulenten Schwankung   \\
 & & \\ 
%--------------------------------------------------------------------------
\\
\newpage \multicolumn{3}{l}{%
\textbf{\textsf{\large Indizes}}
}\\
%-----------------------------------------------
bulk                   & & mittlere Geschwindigkeit \\ 
$i$                    & & Richtungsindex, Spezies \\ 
$j$                    & & Summationsindex, Element \\ 
$n$                    & & Zeitschritt \\
t                      & & turbulent, total, tangential \\
U                      & & Umgebung \\
                       & & \\ %LEERZEILE
%--------------------------------------------------------------------------
%\newpage
\multicolumn{3}{l}{%
\textbf{\textsf{\large Besondere Zeichen}}
}\\
%---------------------------------------------
$\mathrm{d}$ & &      steiles d: vollständiges Differenzial\\  
$\dbar$      & &      palatales d: unvollständiges Differential, alternativ $\delta$\\
$\partial$   & &      partieller Differentialoperator\\
$\Delta$     & &      Differenz\\
 $:=$        & &      definiert durch\\
$\equiv$     & &      identisch\\
$\propto$    & &      proportional\\
$\approx$    & &      etwa \\
$\dot{ x}$	 & &	    Strom von $x$\\
$\bar{ x}$	 & &	    Mittelwert von $x$, oder molare Größe\\
$\hat{ x}$	 & &      Amplitude von $x$\\
(1.1)        & &      Gleichungsnummer, die erste Zahl gibt die\\
             & &         Nummer des Kapitels an, die zweite Zahl ist fortlaufend im Kapitel\\
$[12]$       & &      Nummer im Quellenverzeichnis\\             

 & & \\ % LEERZEILE

%--------------------------------------------------------------------------
\\
\multicolumn{3}{l}{%
\textbf{\textsf{\large Dimensionslose Kennzahlen}}
}\\
%---------------------------------------------
$\mathit{Re} := w L /\nu$  & &   Reynolds-Zahl\\
 & & \\ % LEERZEILE

%---------------------------------------------
\\
\multicolumn{3}{l}{%
\textbf{\textsf{\large Abkürzungen\footnote{Abkürzungen, die bereits im Duden stehen, werden nicht aufgeführt. }}}
}\\
%---------------------------------------------

HsKA     & & Hochschule Karlsruhe -- Technik und Wirtschaft\\
IKKU     & & Institut für Kälte-, Klima- und Umwelttechnik\\
IMP      & & Institute of Materials and Processes\\
MMT      & & Maschinenbau und Mechatronik \\
%  im Duden  PDF      & & Portable Document Format \\ 
R-134a   & & 1,1,1,2-Tetrafluorethan (Kältemittel)\\
RKS      & & Redlich-Kwong-Soave \\
%SST      & & Shear Stress Transport \\
%TKE      & & Turbulente Kinetische Energie \\


\end{longtable}

\cleardoublepage




















        % Formelzeichen- und Abkürzungsverzeichnis 
%
% Inhaltsverzeichnis
%-----------------------------------------------------------------

\manualmark
\markright{Inhaltsverzeichnis.tex}
\markleft{Inhaltsverzeichnis.tex}
\tableofcontents                   % Inhaltsverzeichnis erzeugen
\cleardoublepage




 %
%-------------------------------------------------------------------
%  Hauptteil
%-------------------------------------------------------------------
%
\hypersetup{pageanchor=true}
\include{001_Vorspann/-pagestyle-main}
\chapter{Einleitung}
\label{cha:Einleitung}

\iffalse

%Behauptung
Die kontinuierliche Weiterentwicklung der Gebäudeautomatisierung und die zunehmende Integration von Smart-Home-Systemen haben die Nachfrage nach effizienten und vielseitigen Schalteinrichtungen für die Hausinstallationstechnik erheblich gesteigert. Elektronische Schalter spielen dabei eine Schlüsselrolle, da sie nicht nur die Fernsteuerung elektrischer Lasten ermöglichen, sondern auch dazu beitragen, den Energieverbrauch zu optimieren und die Umweltbelastung zu reduzieren.
Vor diesem Hintergrund ist es von entscheidender Bedeutung, die verschiedenen Ansätze zur Schaltung von \SI{230}{\volt}-Lasten in Hausinstallationssystemen eingehend zu untersuchen.

%Relevanz
Der Bedarf an einer solchen Untersuchung wird durch zwei Hauptfaktoren unterstrichen:
\begin{enumerate}
	\item Die wachsende Nachfrage nach automatisierten Haustechniklösungen, die eine präzise und energieeffiziente Verwaltung elektrischer Lasten ermöglichen.
	\item Die wachsende Bedeutung von Energieeffizienz und Nachhaltigkeit, die den Einsatz effizienter Schaltlösungen unerlässlich macht.
\end{enumerate}

%Thema vorstellen
Diese Bachelorarbeit widmet sich daher einer umfassenden Analyse und dem Vergleich von von elektronischen Schaltertypen, die speziell für die Unterputzmontage in der Hausinstallationstechnik entwickelt wurden. Die betrachteten Schalter umfassen bistabile Relais, MOSFET und Triac.
Dabei wird ein besonderes Augenmerk auf die spezifischen Eigenschaften und Leistungsparameter gelegt, die für die Unterputzmontage von Relevanz sind. Die leitungsgebundenen Störungen, die bei der Verwendung dieser Schalteinrichtungen auftreten können, werden ebenfalls untersucht, um ein umfassendes Verständnis für die praktische Anwendung und die damit verbundenen Herausforderungen zu schaffen.
Zudem werden die verwendeten Schaltungen auf ihre Funktionsweise und verwendeten Schutzmaßnahmen untersucht.

%Ziel vorstellen
Das Ziel dieser Arbeit ist es, eine fundierte Entscheidungsgrundlage für Planer, Entwickler und Endverbraucher zu schaffen, die vor der Wahl der passenden Schalteinrichtung für ihre spezifischen Bedürfnisse stehen. Abschließend werden die gewonnenen Erkenntnisse im Fazit zusammengefasst und ein Ausblick auf zukünftige Entwicklungen und Forschungsfelder im Bereich der Schalteinrichtungen für Unterputzmontage gegeben.

Die Funktionsweise und Bedienung eines Systems mit elektronischen Schaltern wird in \RefFig{fig:System} veranschaulicht. In dieser Darstellung sind die elektronischen Schalteinrichtungen als Elektronik-Einsätze gekennzeichnet, die für das Schalten der angeschlossenen Lasten zuständig sind.
Die Aufsätze auf dem Steckermodul der elektronischen Schalter übertragen die Befehle zum Schalten der angeschlossenen Lasten.
Diese Aufsätze können mit KNX-Funktionalität ausgestattet sein, einem intelligenten Bussystem für die Gebäudesteuerung. Dadurch wird es möglich, Lasten auch dezentral zu steuern und zu schalten, was die Flexibilität und Vielseitigkeit des Systems deutlich erhöht.

\fi
\chapter{Grundlagen}
\label{chap:Grundlagen}

% Unterkapitel einbinden
\section{Stand der Technik bei nicht netzwerkfähigen Feldgeräten}
\label{Stand_Technik_Non_IP}

% Was ich in dem Kapitel sagen will:
% - 



Hier ist Stand der Technik bei nicht netzwerkfähigen 
Feldgeräten \cite{paar_understanding_2024} und \cite{souppaya_nist_2026}


\chapter{Bedrohungsmodell}
\label{Bedrohungsmodell}

Hallo ich bin Bedrohungsmodell.

Und Hier kann ich Sachen hinzufügen.



Test nach Tabelle.  
Hallo, ich schreibe jetzt in VScode.

\emph{Das ist ein doppeltes Enter.}
Das ist ein Enter mit zwei leerzeichen nach dem Punkt.




%
%
%-------------------------------------------------------------------
%Literaturverzeichnis
%-------------------------------------------------------------------
\printbibliography
\addcontentsline{toc}{chapter}{Literaturverzeichnis}
                                \label{cha:LitVerzeichnis}
%\printbibliography[title=Alphabetic]  % alternative (zum Spielen)
%
%-------------------------------------------------------------------
% andere Verzeichnisse
%-------------------------------------------------------------------
%
% Verzeichnisse
%-----------------------------------------------------------------

%\manualmark
%\markright{Tabellenverzeichnis}
%\markleft{ Tabellenverzeichnis}
%\listoftables                      % Tabellenverzeichnis
%      % Die folgende Zeile erzwingt einen Eintag ins Inhaltsverz.
%      \addcontentsline{toc}{chapter}{Tabellenverzeichnis}
%\cleardoublepage

\manualmark
\markright{Abbildungsverzeichnis}
\markleft{ Abbildungsverzeichnis}
\listoffigures                     % Abbildungsverzeichnis erzeugen
      % Die folgende Zeile erzwingt einen Eintag ins Inhaltsverz.
      \addcontentsline{toc}{chapter}{Abbildungsverzeichnis}
\cleardoublepage




 
%
%-------------------------------------------------------------------
% Index
%-------------------------------------------------------------------
%%%\renewcommand{\indexname}{Stichwortverzeichnis}
%%%\printindex
%-------------------------------------------------------------------
%
%
\newpage
%-------------------------------------------------------------------
% Anhang
%-------------------------------------------------------------------
\automark[section]{chapter} % zurücksetzen
\appendix        % Alles was hier kommt, landet im Anhang
%-------------------------------------------------------------------
%
%\chapter{Anhang}
\section{Schaltpläne}
%Schaltpläne
\begin{figure}[H]
	\begin{center}
		\includegraphics[width=.9\textwidth]
		{Anhang/Dateien/MOSFETSchaltplan}
		\caption[Schaltplan der MOSFET-Schaltung]
		{\label{fig:AnhangMOSFET}%
			Schaltplan der MOSFET-Schaltung.}
	\end{center}
\end{figure}

\begin{figure}[H]
	\begin{center}
		\includegraphics[width=1\textwidth]
		{Anhang/Dateien/TriacSchaltplan}
		\caption[Schaltplan der Triac-Schaltung]
		{\label{fig:AnhangTriac}%
			Schaltplan der Triac-Schaltung.}
	\end{center}
\end{figure}

\begin{figure}[H]
	\begin{center}
		\includegraphics[width=1\textwidth]
		{Anhang/Dateien/RelaisSchaltplan}
		\caption[Schaltplan der Relais-Schaltung]
		{\label{fig:AnhangRelais}%
			Schaltplan der Relais-Schaltung.}
	\end{center}
\end{figure}
%
%Testberichte
\section{Prüfungen nach DIN EN 55015}
\includepdf[pages=-]{Anhang/Dateien/MOSFET_L}
\includepdf[pages=-]{Anhang/Dateien/Triac_L}
\includepdf[pages=-]{Anhang/Dateien/Relais_L}





%\include{Anhang/anhang03}


\end{document}   % Nach dieser Zeile darf nichts mehr kommen
%-------------------------------------------------------------------
% E N D E  D E S  D O K U M E N T S
%-------------------------------------------------------------------
