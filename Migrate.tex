Um deine LaTeX-Konfiguration und Pakete auf einen anderen Computer zu spiegeln, kannst du die folgenden Schritte ausführen:

1. LaTeX-Distribution und Pakete
a) Exportiere die installierten Pakete
Falls du TeX Live verwendest, kannst du die Liste der installierten Pakete exportieren. In deinem Workspace sehe ich bereits eine Datei texlive-packages.txt, die vermutlich die Liste der Pakete enthält. Falls diese Datei nicht aktuell ist, kannst du sie mit folgendem Befehl aktualisieren:

b) Pakete auf dem neuen Computer installieren
Kopiere die Datei texlive-packages.txt auf den neuen Computer und führe den folgenden Befehl aus, um die Pakete zu installieren:

Falls du eine andere Distribution wie MikTeX verwendest, kannst du die Pakete manuell installieren oder die Synchronisationsoptionen von MikTeX nutzen.

2. VS Code Konfiguration
a) Einstellungen exportieren
Exportiere deine VS Code-Einstellungen, um die gleiche Konfiguration auf dem neuen Computer zu verwenden:

Öffne die Einstellungen (Ctrl+,).
Klicke oben rechts auf die drei Punkte und wähle Einstellungen als JSON öffnen.
Kopiere den Inhalt der Datei und speichere ihn in einer Datei, z. B. vscode-settings.json.
b) Einstellungen importieren
Auf dem neuen Computer:

Öffne die Einstellungen in VS Code.
Öffne die JSON-Ansicht und füge den Inhalt der vscode-settings.json ein.
c) Erweiterungen synchronisieren
Installiere die gleichen Erweiterungen auf dem neuen Computer. Du kannst die Liste der installierten Erweiterungen exportieren:

Und auf dem neuen Computer installieren:

3. LaTeX Workshop Konfiguration
Falls du die Erweiterung LaTeX Workshop verwendest, überprüfe, ob du benutzerdefinierte Einstellungen hast. Diese sind in der settings.json enthalten (siehe oben). Achte besonders auf:

latex-workshop.latex.tools
latex-workshop.latex.recipes
4. Projektdateien synchronisieren
Da du Git bereits eingerichtet hast, kannst du sicherstellen, dass alle Projektdateien (einschließlich .tex, .bib, .sty, etc.) synchronisiert sind. Stelle sicher, dass .gitignore keine wichtigen Dateien wie .sty-Dateien ausschließt.

5. Testen
Nach der Einrichtung auf dem neuen Computer:

Öffne dein Projekt in VS Code.
Kompiliere eine .tex-Datei, um sicherzustellen, dass alle Pakete und Konfigurationen korrekt funktionieren.
Mit diesen Schritten solltest du deine LaTeX-Umgebung erfolgreich auf den neuen Computer spiegeln können.