%==============================================================
%
% Nomenklatur
%
%==============================================================
\chapter*{Nomenklatur%
  \footnote{ Das ist hier nur beispielhaft; es werden sonst nur (alle) tatsächlich genutzte Zeichen aufgeführt!\\
	beim Sortieren: alphabetisch, erst kleine, dann große Buchstaben.} }%
  \label{Nomenklatur}%
   % Die folgende Zeile erzwingt einen Eintag ins Inhaltsverz.
  \addcontentsline{toc}{chapter}{Nomenklatur}%
%-----------
\manualmark
\markright{Nomenklatur}
\markleft{Nomenklatur}
%\ohead[]{\headmark}
%\begin{longtable}{p{3cm} p{2cm} p{\textwidth}}  % ACHTUNG Breite
\begin{longtable}{p{0.15\textwidth} p{0.15\textwidth} p{0.6\textwidth}}  % ACHTUNG Breite 0.9
%-----------------------------------------------
\multicolumn{3}{l}{%
\textbf{\textsf{\large Lateinische Formelzeichen}}
}\\
%-----------------------------------------------
%
%----Große Buchstaben----
&&\\
%----kleine Buchstaben----
$c\idx{lim}$              & {-}             & Konstante des Turbulenzmodells   \\
$g_i$                     & {m/s$^2$}       & Vektor der Schwerkraft    \\
%$k$                       & {m$^2$/s$^2$}   & turbulente kinetische Energie \\
$k$                       & \si{m^2/s^2}    & turbulente kinetische Energie \\
$l$                       &   m             & Längenmaß \\
$L$                       &  m              & charakteristische Länge \\
$\dot m$                  &   kg/s          & Massenstrom \\
$p$                       &   N/m$^2$       & Druck \\
$R$                       &   -             &  Residuenmatrix   \\
{\textbf{R}}              & \si{J/(mol.K)}  & Universelle Gaskonstante \\
$s$ 										  &  \si{J/(kg.K)}    & spezifische Entropie \\
$ S $                     & J/K             & Entropie \\
$S_{ij}$                  &   1/s           &  Scherrate   \\
$t$                       &   s             & Zeit \\
$ T $                     & K               & Temperatur \\
$u,v,w$                   &   m/s           & Geschwindigkeitskomponente \\
                      & & \\ % LEERZEILE

%----Griechische Buchstaben-----------------------------------------------------------------------
\\
\multicolumn{3}{l}{%
\textbf{\textsf{\large Griechische Formelzeichen}}
 }\\
$\alpha$                &    -                & allgemeine Zustandsgröße \\
$\beta^*$               &    -                & Konstante des Turbulenzmodells \\
$\eta$                  & \si{kg/(m.s)}       & dynamische Viskosität\\  
$\eta$                  &    W/W              & Wirkungsgrad\\  
$\vartheta$             &    \C               & Temperatur\\ 
$\nu$                   &  \si{m^2/s}         & kinematische Viskosität \\
$\varkappa, \kappa$     &     -               & Isentropenexponent \\ 
$\xi$                   &    kg/kg            & Massenanteil\\
$\varrho$               &  \si{kg/m^3}        & Dichte   \\
$\sigma_i$              &    -                & Konstante des Turbulenzmodells \\
$\tau_{ij}$             &  \si{N/m^2}         & Spannungstensor   \\
$\varphi$               &    -                & relative Feuchte \\
$\Phi$                  &    -                & allgemeine Konstante des Turbulenzmodells   \\
$\omega$                &    1/s              & Frequenz der turbulenten Schwankung   \\
 & & \\ 
%--------------------------------------------------------------------------
\\
\newpage \multicolumn{3}{l}{%
\textbf{\textsf{\large Indizes}}
}\\
%-----------------------------------------------
bulk                   & & mittlere Geschwindigkeit \\ 
$i$                    & & Richtungsindex, Spezies \\ 
$j$                    & & Summationsindex, Element \\ 
$n$                    & & Zeitschritt \\
t                      & & turbulent, total, tangential \\
U                      & & Umgebung \\
                       & & \\ %LEERZEILE
%--------------------------------------------------------------------------
%\newpage
\multicolumn{3}{l}{%
\textbf{\textsf{\large Besondere Zeichen}}
}\\
%---------------------------------------------
$\mathrm{d}$ & &      steiles d: vollständiges Differenzial\\  
$\dbar$      & &      palatales d: unvollständiges Differential, alternativ $\delta$\\
$\partial$   & &      partieller Differentialoperator\\
$\Delta$     & &      Differenz\\
 $:=$        & &      definiert durch\\
$\equiv$     & &      identisch\\
$\propto$    & &      proportional\\
$\approx$    & &      etwa \\
$\dot{ x}$	 & &	    Strom von $x$\\
$\bar{ x}$	 & &	    Mittelwert von $x$, oder molare Größe\\
$\hat{ x}$	 & &      Amplitude von $x$\\
(1.1)        & &      Gleichungsnummer, die erste Zahl gibt die\\
             & &         Nummer des Kapitels an, die zweite Zahl ist fortlaufend im Kapitel\\
$[12]$       & &      Nummer im Quellenverzeichnis\\             

 & & \\ % LEERZEILE

%--------------------------------------------------------------------------
\\
\multicolumn{3}{l}{%
\textbf{\textsf{\large Dimensionslose Kennzahlen}}
}\\
%---------------------------------------------
$\mathit{Re} := w L /\nu$  & &   Reynolds-Zahl\\
 & & \\ % LEERZEILE

%---------------------------------------------
\\
\multicolumn{3}{l}{%
\textbf{\textsf{\large Abkürzungen\footnote{Abkürzungen, die bereits im Duden stehen, werden nicht aufgeführt. }}}
}\\
%---------------------------------------------

HsKA     & & Hochschule Karlsruhe -- Technik und Wirtschaft\\
IKKU     & & Institut für Kälte-, Klima- und Umwelttechnik\\
IMP      & & Institute of Materials and Processes\\
MMT      & & Maschinenbau und Mechatronik \\
%  im Duden  PDF      & & Portable Document Format \\ 
R-134a   & & 1,1,1,2-Tetrafluorethan (Kältemittel)\\
RKS      & & Redlich-Kwong-Soave \\
%SST      & & Shear Stress Transport \\
%TKE      & & Turbulente Kinetische Energie \\


\end{longtable}

\cleardoublepage




















