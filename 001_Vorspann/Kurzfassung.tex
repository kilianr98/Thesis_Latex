%%
%%
%%
%\thispagestyle{empty}
%\clearscrheadfoot                  % Alles auf "" setzen
%------------------------------------------------------------------------------

\section*{Sicherer Verbindungsaufbau für nicht netzwerkfähige Feldgeräte auf Basis von Zertifikaten} \label{Kurzfassung}
%Worum geht es
Die steigende Komplexität der Gebäudeautomatisierung und die Integration von Smart-Home-Systemen
haben die Anforderungen an Schalteinrichtungen für die Hausinstallationstechnik deutlich erhöht.
Diese Arbeit bietet eine umfassende Analyse und Vergleich der von der Firma Berker GmbH \& Co. KG speziell für die Unterputzmontage entwickelten elektronischen Schaltertypen.
Hierbei werden Schaltungen mit bistabilen Relais, MOSFETs und Triacs betrachtet.

%Wie wurde vorgegangen
Nach einer einführenden Darstellung der Problemstellung definiert die Arbeit die Anforderungen an \SI{230}{\volt} Schalteinrichtungen,
die sowohl schaltungstechnische als auch wirtschaftliche Aspekte umfassen. Diese Anforderungen werden durch Berechnungen, Simulationen und Messungen vergleichend untersucht. Darüber hinaus wird ein detaillierter Einblick in die Funktionsweise der Schaltungen gegeben, um den Lesern ein umfassendes Verständnis des Themas zu vermitteln.

%Ergebnisse
Die Ergebnisse zeigen, dass das bistabile Relais die besten schalttechnischen Eigenschaften bietet,
jedoch gleichzeitig wirtschaftlich weniger effizient ist. Die Triac-Schaltung ist zwar kostengünstig und einfach,
weist jedoch die schlechtesten elektrischen Eigenschaften auf.
Im Gegensatz dazu bietet die MOSFET-Schaltung einen ausgewogenen Kompromiss zwischen wirtschaftlichen und technischen Aspekten.

%Abschluss
Insgesamt beleuchtet diese Arbeit die komplexen Dynamiken der elektronischen Schalttypen für die Unterputzmontage
und liefert wertvolle Einblicke für Endverbraucher, Installateure und Entwickler im Bereich der Hausinstallationstechnik.
Durch ihre sorgfältige Analyse und den Vergleich trägt die Arbeit dazu bei, das Verständnis für die Leistungsfähigkeit
und Eignung verschiedener Schalttypen in unterschiedlichen Anwendungskontexten zu vertiefen.


\newpage

\section*{Secure Conncetion Establishment for Non-Network-Enabled Field Devices Based on Certitificates} \label{Abstract}
The increasing complexity of building automation and the integration of smart home systems have significantly raised the demands for switchgear in home installation technology. This paper provides a comprehensive analysis and comparison of electronic switch types specifically developed for wall-mounting by Berker GmbH \& Co. KG. These include circuits with bistable relays, MOSFETs, and Triacs.

After an introductory overview of the problem statement, the paper defines the requirements for \SI{230}{\volt} switching devices, encompassing both circuitry and economic aspects. These requirements are comparatively examined through calculations, simulations, and measurements. Furthermore, a detailed insight into the functioning of the circuits is provided to offer readers a comprehensive understanding of the subject.

The results indicate that bistable relays offer the best switching characteristics but are less economically efficient. While the Triac circuit is cost-effective and straightforward, it exhibits the poorest electrical properties. In contrast, the MOSFET circuit presents a balanced compromise between economic and technical aspects.

Overall, this study sheds light on the complex dynamics of electronic switching types for wall-mounted installations and provides valuable insights for end users, installers, and developers in the field of home installation technology. Through its careful analysis and comparison, the study contributes to deepening the understanding of the performance and suitability of various switching types in different application contexts.

\cleardoublepage